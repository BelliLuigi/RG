\section{Lec 14}
Today we will talk about some properties of the Riemann tensor.\par
\[
	[\nabla _{\mu }, \nabla _{\nu }] V^{\rho } = R^{\rho }_{\sigma \mu \nu } V^{\\sigma } + \text{ torsion, so } \nabla _{\rho }V^{\sigma }
\]
and 
\[
	R^{\rho }_{\sigma \mu \nu } = \partial_{\mu }\Gamma^{\rho }_{\nu \sigma } - \partial_{\nu }\Gamma ^{\rho }_{\mu \sigma } + \Gamma  ^{\rho }_{\mu \lambda } \Gamma ^{\lambda }_{\nu \sigma } -\Gamma ^{\rho }_{\nu \lambda }\Gamma ^{\lambda }_{\mu \sigma }
\]
The simplest way to derive the symmetries is to examine the Riemann tensor with all lower indices.
\[
R_{\rho \sigma \mu \nu } = g_{\rho \lambda } R^{\lambda }_{\sigma \mu \nu }
\]
There are four properties that can help us reduce the number of independent entries of this tensor from a number of $n^{4}$ to just 20 (in 4 dimensions spaces).
\begin{enumerate}
\item $R_{\rho \sigma \mu \nu } = - R_{\rho \sigma \nu \mu } $ 
	\item 
	\item 
	\item
\end{enumerate}
So the Riemann tensor is anti-symmetric on the last two indices.\par
We now consider the components of this tensor in locally inertial coordinates $x^{\hat{\mu }}$ at some point \emph{P}. Then the Christoffel symbols will vanish, but not their derivatives.
\begin{equation}
\begin{cases}
g_{\hat{\mu }\hat{\nu }} \left( P \right) = \eta _{\hat{\mu }\hat{\nu }} \\
\partial_{\hat{\rho }} g_{\hat{\mu }\hat{\nu }}\left( P \right) = 0 \to \Gamma ^{\hat{\alpha }}_{\hat{\mu }\hat{\nu }} \left( P \right) = 0 \\
\end{cases}
\end{equation}
We are left with
\begin{gather*}
R_{\hat{\rho }\hat{\sigma }\hat{\mu }\hat{\nu }} \left( P \right) = g_{\hat{\rho }\hat{\lambda }} \left( \partial_{\hat{\mu }} \Gamma ^{\hat{\lambda }}_{\hat{\nu }\hat{\sigma }} - \partial_{\hat{\nu }} \Gamma ^{\hat{\lambda }}_{\hat{\mu }\hat{\sigma }} \right)  = \\
= g_{\hat{\rho }\hat{\lambda }} \frac{1}{2} \partial_{\hat{\mu } } [ g^{\hat{\lambda }\hat{\alpha }} \left( \partial_{\hat{\nu }} g_{\hat{\alpha  }\hat{\sigma }} + \partial_{\hat{\sigma }} g_{\hat{\alpha }\hat{\nu }} - \partial_{\hat{\alpha }} g_{\hat{\sigma }\hat{\nu }} \right) ] - \left( \hat{\mu } \leftrightarrow \hat{\nu } \right) \\
= \frac{g_{\hat{\rho }\hat{\lambda }}}{2} g^{\hat{\lambda }\hat{\alpha }} [ \partial_{\hat{\mu }} \partial_{\hat{\nu }} g_{\hat{\alpha }\hat{\sigma }} + \partial_{\hat{\mu }}\partial_{\hat{\sigma }} g_{\hat{\alpha }\hat{\nu }} - \partial_{\hat{\mu }}\partial_{\hat{\alpha }} g_{\hat{\sigma }\hat{\nu }}] - \left( \hat{\mu } \leftrightarrow \hat{\nu } \right)
\end{gather*} %%% Da rivedere gli indici nella penultima/ultima riga.
Now as usual let's look for some simplifications,



