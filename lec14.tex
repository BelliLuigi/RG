\section{Lec 14}
Today we will talk about some properties of the Riemann tensor.\par
\[
	[\nabla _{\mu }, \nabla _{\nu }] V^{\rho } = R^{\rho }_{\sigma \mu \nu } V^{\\sigma } + \text{ torsion, so } \nabla _{\rho }V^{\sigma }
\]
and 
\[
	R^{\rho }_{\sigma \mu \nu } = \partial_{\mu }\Gamma^{\rho }_{\nu \sigma } - \partial_{\nu }\Gamma ^{\rho }_{\mu \sigma } + \Gamma  ^{\rho }_{\mu \lambda } \Gamma ^{\lambda }_{\nu \sigma } -\Gamma ^{\rho }_{\nu \lambda }\Gamma ^{\lambda }_{\mu \sigma }
\]
The simplest way to derive the symmetries is to examine the Riemann tensor with all lower indices.
\[
R_{\rho \sigma \mu \nu } = g_{\rho \lambda } R^{\lambda }_{\sigma \mu \nu }
\]
There are four properties that can help us reduce the number of independent entries of this tensor from a number of $n^{4}$ to just 20 (in 4 dimensions spaces).
\begin{enumerate}
\item $R_{\rho \sigma \mu \nu } = - R_{\rho \sigma \nu \mu } $ 
	\item 
	\item 
	\item
\end{enumerate}
So the Riemann tensor is anti-symmetric on the last two indices.\par
We now consider the components of this tensor in locally inertial coordinates $x^{\hat{\mu }}$ at some point \emph{P}. Then the Christoffel symbols will vanish, but not their derivatives.
\begin{equation}
\begin{cases}
g_{\hat{\mu }\hat{\nu }} \left( P \right) = \eta _{\hat{\mu }\hat{\nu }} \\
\partial_{\hat{\rho }} g_{\hat{\mu }\hat{\nu }}\left( P \right) = 0 \to \Gamma ^{\hat{\alpha }}_{\hat{\mu }\hat{\nu }} \left( P \right) = 0 \\
\end{cases}
\end{equation}
We are left with
\begin{gather*}
R_{\hat{\rho }\hat{\sigma }\hat{\mu }\hat{\nu }} \left( P \right) = g_{\hat{\rho }\hat{\lambda }} \left( \partial_{\hat{\mu }} \Gamma ^{\hat{\lambda }}_{\hat{\nu }\hat{\sigma }} - \partial_{\hat{\nu }} \Gamma ^{\hat{\lambda }}_{\hat{\mu }\hat{\sigma }} \right)  = \\
= g_{\hat{\rho }\hat{\lambda }} \frac{1}{2} \partial_{\hat{\mu } } [ g^{\hat{\lambda }\hat{\alpha }} \left( \partial_{\hat{\nu }} g_{\hat{\alpha  }\hat{\sigma }} + \partial_{\hat{\sigma }} g_{\hat{\alpha }\hat{\nu }} - \partial_{\hat{\alpha }} g_{\hat{\sigma }\hat{\nu }} \right) ] - \left( \hat{\mu } \leftrightarrow \hat{\nu } \right) \\
= \frac{g_{\hat{\rho }\hat{\lambda }}}{2} g^{\hat{\lambda }\hat{\alpha }} [ \partial_{\hat{\mu }} \partial_{\hat{\nu }} g_{\hat{\alpha }\hat{\sigma }} + \partial_{\hat{\mu }}\partial_{\hat{\sigma }} g_{\hat{\alpha }\hat{\nu }} - \partial_{\hat{\mu }}\partial_{\hat{\alpha }} g_{\hat{\sigma }\hat{\nu }}] - \left( \hat{\mu } \leftrightarrow \hat{\nu } \right)
\end{gather*} 
Now as usual let's look for some simplifications: \par
The first term inside the square brackets is symmetric so it goes away. The product of $g_{\hat{\rho }\hat{\lambda }} g^{\hat{\lambda }\hat{\alpha }} = \delta ^{\hat{\alpha }}_{\hat{\rho }} $ so every $\alpha $ becomes a $\rho $. We are left with
\begin{equation}
	R_{\hat{\rho }\hat{\sigma }\hat{\mu }\hat{\nu }} = \frac{1}{2} [\partial_{\hat{\mu }}\partial_{\hat{\sigma }} g_{\hat{\rho }\hat{\nu }} - \partial_{\hat{\mu }}\partial_{\hat{\rho }}g_{\hat{\sigma }\hat{\nu }} - \partial_{\hat{\nu }}\partial_{\hat{\sigma }} g_{\hat{\rho }\hat{\mu }}+ \partial_{\hat{\nu }}\partial_{\hat{\rho }}g_{\hat{\sigma }\hat{\mu }}]
\end{equation}
By looking at it, we see that the tensor is antisymmetric on it's first two indices
\[
R_{\rho \sigma \mu \nu } = - R_{\sigma \rho \mu \nu }
\]
also for exercise one can see that exchanging block, the tensor is invariant under interchange of the first pair of indices with the second:
\[
R_{\rho \sigma \mu \nu } = R_{\mu \nu \rho \sigma }
\]
another thing that could be checked is that the complete anti-symmetrization of this tensor is null
\[
	R_{[\rho \sigma \mu \nu ]} = 0
\]
so the properties are
\begin{enumerate}
\item $R_{\rho \sigma \mu \nu }  = - R_{\rho \sigma \nu \mu }$
\item $R_{\rho \sigma \mu \nu } = - R_{\sigma \rho \mu \nu }$
\item $R_{\rho \sigma \mu \nu } = R_{\mu \nu \rho \sigma }$
\item $R_{[\rho \sigma \mu \nu ]} = 0$
\end{enumerate}
Now we have to count how many independent entries we are left with.\par
Starting from anti-symmetry on the first two and last two indices and symmetry on the exchange of this pairs, we can think of a symmetric matrix $R_{[\rho \sigma ][\mu \nu ]}$. The pairs are thought as individual indices. An $m\times m$ symmetric matrix has 
\[
\frac{m\left( m+1 \right)}{2}
\]
individual components, while the two anti-symmetric matrices have
\[
\frac{n\left( n-1 \right)}{2}
\]
free components, so
\[
	\frac{1}{2} \left[ \frac{1}{2} n\left( n-1 \right)\right] \left[ \frac{1}{2} n \left( n-1 \right)+1\right] = \frac{1}{8} \left( n^{4} -2n^{3} +3n^{2} -2n \right)
\]
Independent components, and putting $n=4$ we get 21 independent entries. \par
But.\par
We know that a totally antisymmetric tensor with 4 indices has
\[
\frac{n\left( n-1 \right)\left( n-2 \right)\left( n-3 \right)}{4!}
\]
terms and it helps reducing the number of independent components by 1.\par
So, in conclusion the Riemann tensor has 20 free components.\par
\subsubsection{Bianchi Identity}
It's a relation that tells us about the covariant derivative of the Riemann tensor.
There is an algebraic version of the Bianchi Identity that is expressed as
\begin{equation}
R^{\rho }_{\sigma \mu \nu } + R^{\rho }_{\nu \sigma \mu } + R^{\rho }_{\mu \nu \sigma } = 0
\end{equation}
that states that the cyclic permutation of the lower three indices sum to zero.\par
There is a differential way to define the Bianchi Identity that is
\begin{equation}
	\nabla_{[\lambda }R_{\mu \nu ]\rho \sigma } = 0		
\end{equation}

\subsubsection{Ricci tensor}
It is defined as
\begin{equation}
R_{\mu \nu } = R^{\lambda }_{\mu \lambda \nu }
\end{equation}
Is it symmetric? Yes.
\[
R_{\nu \mu } = R^{\lambda }_{ \nu \lambda \mu } = g^{\rho \lambda }R_{\rho \nu \lambda \mu } = g^{\rho \lambda }R_{\lambda \mu \rho \nu } = R^{\lambda }_{\mu \lambda \nu } = R_{\mu \nu }
\]
The trace of the Ricci tensor is the Ricci scalar
\begin{equation}
R = g^{\mu \nu }R_{\mu \nu }
\end{equation}

\subsubsection{A kind of Einstein Equation}
Professor gave us this equation, that is an arrival point
\begin{equation}
R_{\mu \nu } - \frac{1}{2}g_{\mu \nu }R = 8\pi G T_{\mu \nu }
\end{equation}
as we know $T_{\mu \nu }$ describe energy and momentum. We don't want to give up conservation of energy and momentum. 
We know in flat spacetime that
\[
\partial_{\mu }T_{\mu \nu } = 0 
\]
and so in curved spacetime 
\[
\nabla _{\mu }T^{\mu \nu }= 0
\]
Now we need a tensor in order to keep it null when we change frame. This means that also in the EE we need 
\[
\nabla _{\mu }\left( G_{\mu \nu } \right) = 0
\]
where the tensor $G_{\mu \nu }$ called Einstein tensor is defined as
\begin{equation}
g_{\mu \nu } = R_{\mu \nu } - \frac{1}{2} g_{\mu \nu }R
\end{equation}
let's prove this.
\begin{gather*}
	g^{\mu \lambda } g^{\nu \sigma } \left[ \nabla _{\lambda } R_{\rho \sigma \mu \nu } + \nabla _{\rho }R_{\sigma \lambda \mu \nu } + \nabla _{\sigma }R_{\lambda \rho \mu \nu }\right] = 0 \\
	\nabla ^{\mu }R_{\rho \mu } - \nabla _{\rho }R + \nabla ^{\mu }R_{\rho \nu }  = 0\\
	\text{ or } \nabla ^{\mu }R_{\rho \mu } = \frac{1}{2} \nabla _{\rho }R
\end{gather*}

































