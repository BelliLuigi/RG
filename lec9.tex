\section{Lec 9}
\subsubsection{Active and passive transformations}
Active transformations change the physical position of a set of points relative to a fixed coordinate system. \par
Passive transformations leave the points fixed but change the coordinate system relative to which they are described.
We prefer the first approach, we leave vectors untouched and transform RFs.

\subsection{Still on Metric Tensor}
We described a little this tensor in the previous lecture. I remember that:
\[
det\left( g \right) = g \neq 0
\]
in 3D Euclidean Space
\[
	g = \mathbb{I} = \begin{pmatrix}
	1 & 0 & 0 \\
	0 & 1 & 0 \\
	0 & 0 & 1
	\end{pmatrix} 		
\]
and in spherical it would be like
\[
\begin{pmatrix}
1 & 0 & 0 \\
0 & r^{2}sin^{2}\theta  & 0 \\
0 & 0 & r^{2}
\end{pmatrix} 
\]
In general the metric tensor has this form
\[
g_{\mu \nu } = diag\left( -1, \ldots, 1, \ldots , 0, \ldots   \right)
\]
If there are just '$+1$', the metric is Euclidean.\par
If I have one ' $-1$' and only ' $+1$' the metric is Lorentzian.\par
We will not discuss other combinations.

Today we will try to formalize EEP.\par

Let $P$ be a spacetime point and 
\[
g_{\mu \nu }\left( P \right)= \text{ some generic matrix } \neq \eta_{\mu \nu }
\]
I want to find new coordinates $x^{\hat{\mu }}$ such that $g_{\hat{\mu }\hat{\nu }}\left( P \right)=\eta_{\mu \nu }$ and $\partial_{\hat{\rho }}g_{\hat{\mu }\hat{\nu }}\left( P \right)=0$.
