\section{Lec 8}
\subsection{Brief Recap}
We saw the WEP, that states $m_{i}= m_{g}$, and as consequence we get that is is impossible distinguish a gravitational field from motion, at least locally. \par
With the EEP we were able to derive the expression that quantify the gravitational redshift.
The SEP included gravity.\par

Focusing on EEP, we introduced \emph{Locally Inertial Frames, LIF} (equivalent to Freely Falling Frames).
Having accepted that gravity cannot be treated as a force, because it is impossible to disentangle acceleration due to gravity, we identified a preferred class of frames: LIFs.\par
In LIFs laws of physics are equal to the laws of SR and spacetime is\break Minkowskian.

We introduced Coordinates: with a generic set M, a chart, given a subset $U \subset M $, is a injective linear  map $\phi $, that \[
\phi : U \to \mathbb{R}^{n}
\]
An \emph{atlas }is an indexed $\{U_{\alpha}, \phi_{\alpha }\}$, in such a way that $U_{\alpha }$ cover \emph{M}.
A \emph{manifold} is a set M along with an atlas. A manifold can be $C^{\infty}$ if $\phi $ are differentiable an infinite amount of times, otherwise is $C^{p}$, differentiable p-times.
\subsubsection{Vectors again again}
We resurrected $T_{P}$, \emph{P} generic \emph{M} point. $T_{P}$ is the vector space of all the vectors defined at that point.
\begin{quote}
$T_{P}$ is identified with the space of directional derivatives operators acting along the curves through \emph{P}
\end{quote}
Why this identification makes sense?\par
A generic curve through spacetime, that we call WL, is a parametric curve that is indicated by $x^{\mu }\left( \lambda  \right)$: for a specific value of $\lambda $ I have the $x^{\mu }$ point.
\[
x^{\mu }\left( \overline{\lambda } \right) = P
\]
How do I define the directional derivatives?
Be \[
\frac{d}{d\lambda }
\]
that acts on functions, \emph{f},
\[
f : M \to \mathbb{R}
\]
then
\begin{equation}
\frac{d}{d\lambda } \left( f \right) = \frac{\partial f}{\partial x^{\mu }} \frac{d x^{\mu }}{d \lambda } = \frac{dx^{\mu }}{d\lambda } \frac{\partial}{\partial x^{\mu }} f
\end{equation}
And if true $\forall f$, I can identify the equality among the operators:
\[
\frac{d}{d\lambda } = \frac{dx^{\mu }}{d\lambda }\partial_{\mu }
\]
We see that it is like a basis.\par
\textbf{Basis vectors } for $T_{P}\to \partial_{\mu }$. (we previously called them $\hat{e}_{\left( \mu  \right)}$). \par
In conclusion a generic vector is 
\[
V = V^{\mu }\partial_{\mu }
\]
where $V^{\mu }$ are the components, and $\partial_{\mu }$ are the basis elements.\par

It's very easy to show how vectors transform because we know how derivatives transform:
\begin{equation}
\frac{\partial}{\partial  x^{\mu'}} = \frac{\partial  x^{\mu }}{\partial  x^{\mu '}} \frac{\partial}{\partial  x^{\mu }}
\end{equation}
or in a tensor-like notation:
\begin{equation}
\partial_{\mu '} = \frac{\partial  x^{\mu }}{\partial x^{\mu '}} \partial_{\mu }
\end{equation}
If \emph{V} tensor is invariant, by definition, it's components transform anyway like
\[
V^{\mu '} = \frac{\partial x^{\mu '}}{\partial x^{\mu }}V^{\mu }
\]

\paragraph{Example} LTs
\[
x^{\alpha '} = \Lambda^{\alpha '}_{\alpha }x^{\alpha }
\]
I consider a specific change of coordinates: LTs, so
\[
\frac{\partial x^{\mu '}}{\partial x^{\mu }} = \frac{\partial}{\partial x^{\mu }} \left( \Lambda^{\mu '}_{\alpha }x^{\alpha } \right) = \Lambda^{\mu '}_{\alpha } \frac{\partial x^{\alpha }}{\partial x^{\mu }} = \Lambda^{\mu '}_{\alpha } \delta^{\alpha }_{\mu } = \Lambda^{\mu '}_{\mu }
\]
we get under more general transformations vectors components transform like this, and more, we recovered the LT transformation.

\subsection{Dual Vectors}
Be the \emph{cotangent space} $T_{P}^{*}$. If $\omega \in T_{P}^{*}$, $\omega $is a linear map $\omega : T_{P}\to \mathbb{R}$.
We want to define formally $T_{P}^{*}$ (like we did for $T_{P}$).
We know that the tangent space $T_{P}$ holds directional derivatives, while cotangent space $T_{P}^{*}$ holds gradients. \par
For a generic $f: M\to \mathbb{R}$:
\begin{gather*}
\frac{d}{d\lambda } \in T_{P} \\
d \in T_{P}^{*} \\
\text{ and so } \\
df \left( \frac{d}{d\lambda } \right) \equiv \frac{df}{d\lambda }\\
\downarrow \quad\; \downarrow \qquad \;\downarrow \\
\in T_{P}^{*}, \in T_{P}, \in \mathbb{R}
\end{gather*}
\textbf{Basis} for T\textsubscript{P}\textsuperscript{*}: $dx^{\mu }$.\par
\[
dx^{\mu } \left( \frac{\partial}{\partial x^{\nu }} \right) = \frac{\partial x^{\mu }}{\partial x^{\nu }} = \delta^{\mu }_{\nu }
\]
that is the same as the old $\hat{O}^{\left( \mu  \right)}\left( \hat{e}_{\nu } \right)=\delta^{\mu }_{\nu }$.

A dual vector is \[
\omega = \omega_{\mu }dx^{\mu }
\]
the basis component transform like
\[
dx^{\mu '} = \frac{\partial x^{\mu '}}{\partial x^{\mu }} dx^{\mu }
\]
and the vector components
\[
\omega_{\mu '} = \frac{\partial x^{\mu }}{\partial x^{\mu '}}\omega_{\mu }
\]

\subsection{Tensors (k,l)}
\[
T : T_{P}^{*}\times \ldots \times T_{P}^{*} \times T_{P}\times \ldots \times T_{P}\to \mathbb{R}
\]
Tensors can be expanded into components:
\[
T = T^{\mu_{1}\ldots \mu_{k}}_{\nu_{1}\ldots \nu_{l}} \left( \partial_{\mu_{1}} \otimes \ldots \otimes \partial_{\mu_{k}} \otimes dx^{\nu_{1}} \otimes \ldots \otimes dx^{\nu_{l}} \right)
\]
So the components of a generic tensor transform like
\[
T^{\mu_{1}' \ldots \mu_{k}'}_{\nu_{1}' \ldots \nu_{l}'} = \left( \frac{\partial x^{\mu_{1}'}}{\partial x^{\mu_{1}}} \ldots \frac{\partial x^{\nu_{1}}}{\partial x^{\nu_{1}'}}\ldots  \right) T^{\mu_{1}\ldots \mu_{k}}_{\nu_{1}\ldots \nu_{l}}
\]

Now let's see a unusual tensor, it's a (2,1) tensor, how does transform?
\[
T^{\alpha' \beta'}_{\gamma'} = \frac{\partial x^{\alpha '}}{\partial x^{\alpha }} \frac{\partial x^{\beta '}}{\partial x^{\beta }} \frac{\partial x^{\gamma }}{\partial x^{\gamma '}} T^{\alpha \beta }_{\gamma }	 
\]
\paragraph{Example/exercise} (from 2.4 of Carroll)
Be a tensor $S_{ij}$, with $i,j =1,2$, so it's a (0.2) tensor.
We know that
\begin{itemize}
	\item S\textsubscript{11} = 1
	\item S\textsubscript{12}=S\textsubscript{21} = 0
	\item S\textsubscript{22} = x\textsuperscript{2}
\end{itemize}
We get new coordinates:
\begin{gather*}
x' = \frac{2x}{y} \\
y' = \frac{y}{2}
\end{gather*}
What are the expressions for $S_{i'j'}$?\par
One could think to compute each entry doing
\[
S_{i'j'}= \frac{\partial x^{i}}{\partial x^{i'}} \frac{\partial x^{j}}{\partial x^{j'}} S_{ij}
\]
So, like the (1,1) one looks like
\[
S_{1'1'} = \frac{\partial x^{i}}{\partial x^{1'}} \frac{\partial x^{j}}{\partial x^{1'}} S_{ij}
\]
But it seems that there is a much faster way than this.
