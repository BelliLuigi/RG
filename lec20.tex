\chapter{Gravitational Waves}
\section{Linearized gravity}
Roughly speaking this is gravitational waves, while the macro area is \emph{linearized gravity} which is the study of GR where $g_{\mu \nu }$ can be decomposed in
\[
g_{\mu \nu } = \eta _{\mu \nu }+ h_{\mu \nu }, \left( |h_{\mu \nu }|<<1 \right)
\]
and $h_{\mu \nu }$ is a symmetric matrix.
That's nothing new, right? What we assumed back then was
\begin{itemize}
\item weak field
\item motion of slow test particle\footnote{A \emph{test particle} is and idealized concept to study the properties of spacetime and gravitational fields without significantly disturbing them. Specifically
	\begin{itemize}
	\item Infinitesimally small, point particle
	\item Negligible mass and energy, no contribute to curvature of spacetime.
	\item Moves along geodesics
	\item Used to explore effects of spacetime curvature.
	\end{itemize}
	}
\item static field.
\end{itemize}
We will give up on the last two assumptions for our derivation.\par
As just said, we  will start from
\begin{equation}
g_{\mu \nu } = \eta _{\mu \nu } + h_{\mu \nu } 
\end{equation}
we want to find the \emph{inverse metric} $g^{\mu \nu }$, we know
\[
g^{\mu \nu }g_{\nu \rho  } = \delta ^{\mu }_{\rho }
\]
let's express $g^{\mu \nu }$ like
\begin{equation}\label{eq:inversemetric}
g^{\mu \nu } = \eta ^{\mu \nu } + \Delta ^{\mu \nu }
\end{equation}
we will keep from now on only terms that are linear in $h_{\mu \nu }$, so $O\left( h \right)$ or that have the same order of $h_{\mu \nu }$.\par
\begin{align}
	g^{\mu \nu }g_{\nu \rho } &= \delta ^{\mu }_{\rho }\\
	\left( \eta ^{\mu \nu } +\Delta ^{\mu \nu } \right)\left( \eta _{\nu \rho }+h_{\nu \rho } \right) &= \delta ^{\mu }_{\rho }\\
	\delta ^{\mu }_{\rho } + \eta ^{\mu \nu }h_{\nu \rho } + \Delta ^{\mu \nu }\eta _{\nu \rho } + O\left( h^{2} \right) &= \delta ^{\mu }_{\rho } \\
	\eta ^{\mu \nu }h_{\nu \rho } + \Delta ^{\mu \nu }\eta _{\nu \rho } & = 0 \\
	\eta ^{\alpha \rho }\Delta ^{\mu \nu }\eta _{\nu \rho } & = - \eta ^{\mu \nu }h_{\nu \rho }\eta ^{\alpha \rho } \\
	\delta ^{\alpha }_{\nu }\Delta ^{\mu \nu } &= - \eta ^{\mu \nu }\eta ^{\alpha \rho }h_{\nu \rho } \\
	\Delta ^{\mu \alpha } &= -\eta ^{\mu \nu }\eta ^{\alpha \rho }h_{\nu \rho } \\
	\Delta ^{\mu \nu } &= -\eta ^{\mu \alpha }\eta ^{\nu \beta }h_{\alpha \beta }
\end{align}
in the last step we changed indices to insert it on eq.\ref{eq:inversemetric}:
\begin{equation}
g^{\mu \nu } = \eta ^{\mu \nu } + \Delta^{\mu \nu } = \eta ^{\mu \nu }- \eta ^{\mu \alpha }\eta ^{\nu \beta }h_{\alpha \beta }
\end{equation}
The theory we are studying is of a dynamical symmetric tensor that propagates in flat spacetime, where the word \emph{propagates} gives us an hint about our goal. This is to take take \emph{h} and rewrite the Einstein Equation that is a equation of motion. \par
How does \emph{h} transform under LTs?
\begin{align}
	g_{\mu ^{\prime }\nu ^{\prime }} &= \Lambda ^{\mu }_{\mu ^{\prime }} \Lambda ^{\nu }_{\nu ^{\prime }}g_{\mu \nu } = \\
	&= \Lambda ^{\mu }_{\mu ^{\prime }}\Lambda ^{\nu }_{\nu ^{\prime }}\left( \eta _{\mu \nu }+h_{\mu \nu } \right) = \\
	&= \eta _{\mu ^{\prime }\nu ^{\prime }} + \Lambda ^{\mu }_{\mu ^{\prime }}\Lambda ^{\nu }_{\nu ^{\prime }}h_{\mu \nu } = \\
	&= \eta _{\mu ^{\prime }\nu ^{\prime }}+h_{\mu ^{\prime }\nu ^{\prime }}
\end{align}
\paragraph{Christoffel symbol}
Now we can compute the Christoffel symbol
\begin{align}
	\Gamma ^{\rho }_{\mu \nu } &= \frac{1}{2} g^{\rho \sigma }\left[ \partial_{\mu }g_{\sigma \nu } + \partial_{\nu }g_{\sigma \mu } - \partial_{\sigma }g_{\mu \nu } \right] =\\
	&= \frac{1}{2} \eta ^{\rho \sigma }\left[ \partial_{\mu }h_{\sigma \nu }+\partial_{\nu }h_{\sigma \mu } - \partial_{\sigma }h_{\mu \nu } \right]
\end{align}
we kept just $\eta $ outside the square brackets because the partial derivatives inside them are already $O\left( h \right)$, since derivative of $\eta $ is null.\par
\paragraph{Riemann tensor}
\begin{align}
	R^{\rho }_{\sigma \mu \nu } &= \partial_{\mu }\Gamma ^{\rho }_{\sigma \nu } - \partial_{\nu }\Gamma ^{\rho }_{\sigma \mu } + \Gamma \Gamma -\Gamma \Gamma  = \\
				    &= \frac{1}{2} \partial_{\mu } \left[ \eta ^{\rho \alpha }\left( \partial_{\sigma} h_{\alpha \nu }+ \partial_{\nu }h_{\alpha \sigma }- \partial_{\alpha }h_{\sigma \nu } \right) \right] - \frac{1}{2}\partial_{\nu }\left[ \eta ^{\rho \alpha }\left( \partial_{\sigma }h_{\alpha \mu } +\partial_{\mu }h_{\sigma \alpha }- \partial_{\alpha }h_{\sigma \mu } \right) \right] 
\end{align}
terms that are $\Gamma \Gamma $ as you can see are not linear in $h$.\par
Now we will lower the upper index of the Riemann tensor since it is easier to manipulate.
























