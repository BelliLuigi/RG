\chapter{Gravitational Waves}
\section{Linearized gravity}
Roughly speaking this is gravitational waves, while the macro area is \emph{linearized gravity} which is the study of GR where $g_{\mu \nu }$ can be decomposed in
\[
g_{\mu \nu } = \eta _{\mu \nu }+ h_{\mu \nu }, \left( |h_{\mu \nu }|<<1 \right)
\]
and $h_{\mu \nu }$ is a symmetric matrix.
That's nothing new, right? What we assumed back then was
\begin{itemize}
\item weak field
\item motion of slow test particle\footnote{A \emph{test particle} is and idealized concept to study the properties of spacetime and gravitational fields without significantly disturbing them. Specifically
	\begin{itemize}
	\item Infinitesimally small, point particle
	\item Negligible mass and energy, no contribute to curvature of spacetime.
	\item Moves along geodesics
	\item Used to explore effects of spacetime curvature.
	\end{itemize}
	}
\item static field.
\end{itemize}
We will give up on the last two assumptions for our derivation.\par
As just said, we  will start from
\begin{equation}
g_{\mu \nu } = \eta _{\mu \nu } + h_{\mu \nu } 
\end{equation}
we want to find the \emph{inverse metric} $g^{\mu \nu }$, we know
\[
g^{\mu \nu }g_{\nu \rho  } = \delta ^{\mu }_{\rho }
\]
let's express $g^{\mu \nu }$ like
\begin{equation}\label{eq:inversemetric}
g^{\mu \nu } = \eta ^{\mu \nu } + \Delta ^{\mu \nu }
\end{equation}
we will keep from now on only terms that are linear in $h_{\mu \nu }$, so $O\left( h \right)$ or that have the same order of $h_{\mu \nu }$.\par
\begin{align}
	g^{\mu \nu }g_{\nu \rho } &= \delta ^{\mu }_{\rho } \nonumber\\
	\left( \eta ^{\mu \nu } +\Delta ^{\mu \nu } \right)\left( \eta _{\nu \rho }+h_{\nu \rho } \right) &= \delta ^{\mu }_{\rho } \nonumber\\
	\delta ^{\mu }_{\rho } + \eta ^{\mu \nu }h_{\nu \rho } + \Delta ^{\mu \nu }\eta _{\nu \rho } + O\left( h^{2} \right) &= \delta ^{\mu }_{\rho } \nonumber\\
	\eta ^{\mu \nu }h_{\nu \rho } + \Delta ^{\mu \nu }\eta _{\nu \rho } & = 0 \nonumber\\
	\eta ^{\alpha \rho }\Delta ^{\mu \nu }\eta _{\nu \rho } & = - \eta ^{\mu \nu }h_{\nu \rho }\eta ^{\alpha \rho } \nonumber\\
	\delta ^{\alpha }_{\nu }\Delta ^{\mu \nu } &= - \eta ^{\mu \nu }\eta ^{\alpha \rho }h_{\nu \rho } \nonumber\\
	\Delta ^{\mu \alpha } &= -\eta ^{\mu \nu }\eta ^{\alpha \rho }h_{\nu \rho } \nonumber\\
	\Delta ^{\mu \nu } &= -\eta ^{\mu \alpha }\eta ^{\nu \beta }h_{\alpha \beta }
\end{align}
in the last step we changed indices to insert it on eq.\ref{eq:inversemetric}:
\begin{equation}
g^{\mu \nu } = \eta ^{\mu \nu } + \Delta^{\mu \nu } = \eta ^{\mu \nu }- \eta ^{\mu \alpha }\eta ^{\nu \beta }h_{\alpha \beta }
\end{equation}
The theory we are studying is of a dynamical symmetric tensor that propagates in flat spacetime, where the word \emph{propagates} gives us an hint about our goal. This is to take take \emph{h} and rewrite the Einstein Equation that is a equation of motion. \par
How does \emph{h} transform under LTs?
\begin{align}
	g_{\mu ^{\prime }\nu ^{\prime }} &= \Lambda ^{\mu }_{\mu ^{\prime }} \Lambda ^{\nu }_{\nu ^{\prime }}g_{\mu \nu } = \nonumber\\
	&= \Lambda ^{\mu }_{\mu ^{\prime }}\Lambda ^{\nu }_{\nu ^{\prime }}\left( \eta _{\mu \nu }+h_{\mu \nu } \right) = \nonumber\\
	&= \eta _{\mu ^{\prime }\nu ^{\prime }} + \Lambda ^{\mu }_{\mu ^{\prime }}\Lambda ^{\nu }_{\nu ^{\prime }}h_{\mu \nu } = \nonumber\\
	&= \eta _{\mu ^{\prime }\nu ^{\prime }}+h_{\mu ^{\prime }\nu ^{\prime }}
\end{align}
\paragraph{Christoffel symbol}
Now we can compute the Christoffel symbol
\begin{align}
	\Gamma ^{\rho }_{\mu \nu } &= \frac{1}{2} g^{\rho \sigma }\left[ \partial_{\mu }g_{\sigma \nu } + \partial_{\nu }g_{\sigma \mu } - \partial_{\sigma }g_{\mu \nu } \right] = \nonumber\\
	&= \frac{1}{2} \eta ^{\rho \sigma }\left[ \partial_{\mu }h_{\sigma \nu }+\partial_{\nu }h_{\sigma \mu } - \partial_{\sigma }h_{\mu \nu } \right]
\end{align}
we kept just $\eta $ outside the square brackets because the partial derivatives inside them are already $O\left( h \right)$, since derivative of $\eta $ is null.\par
\paragraph{Riemann tensor}
\begin{align}
	R^{\rho }_{\sigma \mu \nu } = \partial_{\mu }\Gamma ^{\rho }_{\sigma \nu } - \partial_{\nu }\Gamma ^{\rho }_{\sigma \mu } + \Gamma \Gamma -\Gamma \Gamma  = \nonumber\\
				    = \frac{1}{2} \partial_{\mu } \left[ \eta ^{\rho \alpha }\left( \partial_{\sigma} h_{\alpha \nu }+ \partial_{\nu }h_{\alpha \sigma }- \partial_{\alpha }h_{\sigma \nu } \right) \right] - \nonumber\\
				     \frac{1}{2}\partial_{\nu }\left[ \eta ^{\rho \alpha }\left( \partial_{\sigma }h_{\alpha \mu } +\partial_{\mu }h_{\sigma \alpha }- \partial_{\alpha }h_{\sigma \mu } \right) \right] \nonumber
\end{align}
terms that are $\Gamma \Gamma $ as you can see are not linear in $h$.\par
Now we will lower the upper index of the Riemann tensor since it is easier to manipulate.
\begin{gather}
	R_{\lambda \sigma \mu \nu } \simeq \eta _{\lambda \rho }R^{\rho }_{\sigma \mu \nu } = \nonumber \\
	 =  \eta _{\lambda \rho }\eta ^{\rho \alpha } \left( \partial_{\mu }\partial_{\sigma }h _{\alpha \nu } + \partial_{\mu }\partial_{\nu }h_{\alpha \sigma }-\partial_{\mu }\partial_{\alpha }h_{\sigma \nu } \right) - \nonumber\\
	  \frac{1}{2}\eta _{\lambda \rho }\eta ^{\rho \alpha }\left( \partial_{\nu }\partial_{\sigma }h_{\alpha \mu } + \partial_{\nu }\partial_{\mu }h_{\sigma \alpha }-\partial_{\nu }\partial_{\alpha }h_{\sigma \mu } \right) =  \nonumber\\
	  = \frac{1}{2} \delta ^{\alpha }_{\lambda }\left[ \left( \ldots  \right)\left( \ldots  \right) \right] = \nonumber\\
	  = \frac{1}{2} \left[ \left( \partial_{\mu }\partial_{\sigma }h_{\lambda \nu } +\partial_{\mu }\partial_{\nu }h_{\lambda \sigma }-\partial_{\nu }\partial_{\lambda }h_{\sigma \nu } \right) - \left( \partial_{\nu }\partial_{\sigma }h_{\lambda \mu }+ \partial_{\nu }\partial_{\mu }h_{\sigma \lambda } - \partial_{\nu }\partial_{\lambda }h_{\sigma \mu } \right) \right] \nonumber\\
	 \text{ {\small Partial derivatives commute, \emph{h} is symmetric }} \nonumber\\
	 = \frac{1}{2} \left[ \partial_{\mu }\partial_{\sigma }h_{\lambda \nu }-\partial_{\mu }\partial_{\lambda }h_{\sigma \nu } - \partial_{\nu }\partial_{\sigma }h_{\lambda \mu } + \partial_{\nu }\partial_{\lambda }h_{\sigma \mu } \right]
\end{gather}
\paragraph{Ricci tensor}

\begin{gather}
R^{\mu }_{\sigma \mu \nu } = \frac{1}{2}\left[ \partial^{\mu }\partial_{\sigma }h_{\mu \nu }- \partial^{\mu }\partial_{\mu }h_{\sigma \nu } - \partial_{\nu }\partial_{\sigma }h^{\mu }_{\mu }+ \partial_{\nu }\partial^{\mu }h_{\sigma \mu } \right] \nonumber\\
R_{\sigma \nu } = \frac{1}{2} \left( \partial^{\mu }\partial_{\sigma }h_{\mu \nu } + \partial^{\mu }\partial_{\nu  }h_{\mu \sigma }+ \partial_{\sigma }\partial_{\nu }h-\Box h_{\sigma \nu } \right)\label{eq:lingriccitns}
\end{gather}
with
\begin{gather*}
h \equiv \eta ^{\mu \nu }h_{\mu \nu }, \text{ the trace of \emph{h} } \\
\Box = \partial_{\mu }\partial^{\mu } = -\partial^{2}_{t} + \partial^{2}_{x}+\partial^{2}_{y}+\partial^{2}_{z}
\end{gather*}
\paragraph{Ricci scalar}
\begin{equation}
R = \eta ^{\sigma \nu }R_{\sigma \nu } = \frac{1}{2}\left( \partial^{\mu }\partial^{\sigma }h_{\mu \sigma }+ \partial^{\mu }\partial^{\sigma }h_{\sigma \mu }-\Box h -\Box h \right)= \partial^{\mu }\partial^{\sigma }h_{\mu \sigma }-\Box h
\end{equation}

\subsection{Gauge Invariant}
In the context of linearized gravity, gauge invariance refers to the freedom to make certain transformations to the metric perturbation $h_{\mu \nu }$ without changing the physical content of the theory. In the linearized regime, this invariance implies that $h_{\mu \nu }$ can be transformed under an infinitesimal coordinate change
\[
x^{\mu } \to x^{\mu }+ \xi^{\mu }  
\]
$\xi $ is dependent on coordinates, is not a constant, and is small.\par
Let's see how we write the metric tensor in new coordinate system.
\begin{align}
	g_{\mu ^{\prime }\nu ^{\prime }} &= \frac{\partial x^{\mu }}{\partial x^{\mu ^{\prime }}} \frac{\partial x^{\nu }}{\partial x^{\nu ^{\prime }}} \left( \eta _{\mu \nu } +h_{\mu \nu } \right) = \nonumber\\
					 &= \left( \delta ^{\mu }_{\mu ^{\prime }} + \partial_{\mu ^{\prime }}\xi ^{\mu } \right)\left( \partial^{\nu }_{\nu ^{\prime }}+\xi ^{\nu } \right)\left( \eta _{\mu \nu }+h_{\mu \nu } \right) = \nonumber\\
	&= \eta _{\mu ^{\prime }\nu ^{\prime }} + h_{\mu ^{\prime }\nu ^{\prime }} +\partial_{\mu ^{\prime }}\xi _{\nu ^{\prime }}+\partial_{\nu ^{\prime }}\xi _{\mu ^{\prime }}
\end{align}
where the last three terms are
\begin{equation}
\partial^{\mu }_{\mu ^{\prime }}\partial^{\nu }_{\nu ^{\prime }}h_{\mu \nu } \equiv   h_{\mu ^{\prime }\nu ^{\prime }} +\partial_{\mu ^{\prime }}\xi _{\nu ^{\prime }}+\partial_{\nu ^{\prime }}\xi _{\mu ^{\prime }}
\end{equation}
we did this infinitesimal change of coordinates and we found that the metric is minkowskian plus a correction.\par
We have
\[
\begin{rcases*}
h_{\mu \nu }\\
h_{\mu \nu }+ 2\partial_{(\mu  }\xi _{\nu )}
\end{rcases*}
\text{ that are equivalent }
\]
they both describe the same physical perturbation. Indeed \emph{gauge invariance} allows $h_{\mu \nu } $ to be shifted as
\[
h_{\mu \nu }\to h_{\mu \nu }+\partial_{\mu }\xi _{\nu }+\partial_{\nu }\xi _{\mu }
\]
Now we can compute the variation of the linearized Riemann tensor under this transformation
\begin{gather*}
	\delta R_{\lambda \sigma \mu \nu } = \frac{1}{2}\left[ \partial_{\mu }\partial_{\sigma }\delta h_{\nu \lambda } - \partial_{\mu }\partial_{\lambda }\delta h_{\sigma \nu } - \partial_{\nu }\partial_{\sigma }\delta h_{\lambda \mu } + \partial_{\nu }\partial_{\lambda }\delta h_{\sigma \mu } \right]\\
	= \frac{1}{2}[ \colorbox{yellow}{\underline{$ \partial_{\mu }\partial_{\sigma }$}$( \partial_{\nu }\xi _{\lambda }$}+ \text{ \underline{$\partial_{\lambda }\xi _{\nu }$}} ) \colorbox{pink}{\underline{$-  \partial_{\mu }\partial_{\lambda }$}}\text{\underline{$( \partial_{\sigma }\xi _{\nu }$}}+ \colorbox{pink}{$ \partial_{\nu }\xi _{\sigma }$}) + \\
	\colorbox{yellow}{$ - \partial_{\nu }\partial_{\sigma }$}\left( \partial_{\lambda }\xi _{\mu }+ \colorbox{yellow}{$ \partial_{\mu }\xi _{\lambda } $}\right) + \colorbox{pink}{$ \partial_{\nu }\partial_{\lambda }$}\left( \partial_{\sigma }\xi _{\mu }+ \colorbox{pink}{$ \partial_{\mu }\xi _{\sigma }$} \right) ] 
\end{gather*}
As you see by colors and underlining each term cancel with an opposite other. Since the variation of the Riemann tensor is null, this transformation leaves Riemann tensor, Ricci tensor and scalar unchanged.\par

If you have a gauge transformation, which is defined as the actual \emph{h}, nothing changes at the level of the Riemann tensor. 
\subsection{Degrees of Freedom for $h_{\mu \nu }$}
I can choose $\xi $ in such a way that $h$ takes the form that is particularly convenient for our purposes. Let's see more deeply the meaning of $h_{\mu \nu }$ tensor and its components.\par
The issue of gauge invariant will be postponed to the next lecture.\par
We assumed we work on a given coordinates system, we pick a specific gauge, and we will discuss what happens to $h_{\mu \nu }$.\par
As one may remember
\[
h_{\mu ^{\prime }\nu ^{\prime }} = \Lambda ^{\mu }_{\mu ^{\prime }} \Lambda ^{\nu }_{\nu ^{\prime }}h_{\mu \nu }
\]
the components of $h$, transform exactly as a Lorents tensor. I change coordinates and this is a LT. Or I could say that the tensor transform as a Lorentz tensor, this sentence is stronger than the previous, and we already checked it. In particular when we discuss LTs we have rotations (3) and boosts (3). Let's focus on the rotations. It's convenient to decompose the metric $h$:
\[
h_{\mu \nu } = \begin{pmatrix}
h_{00} & h_{01} & h_{02} & h_{03} \\
h_{10}=h_{01} & \ldots  & \ldots  & \ldots  \\
h_{02} & \ldots  & \ldots  & \ldots  \\
h_{03} & \ldots  & \ldots  & \ldots 
\end{pmatrix} 
\]
because the $h_{\mu \nu }$ tensor is symmetric.\par
Under \textbf{spatial rotations}:
\begin{itemize}
\item $h_{00}$ is a scalar, and it's invariant for spatial rotations, and $h_{00} = -2\phi $, (1 degree of freedom)
\item $h_{0i}$ is a vector, (3 degrees of freedom), $h_{0i} = w_{i}$
\item $h_{ij}$ is a symmetric rank-2 tensor, (6 degrees of freedom), $h_{ij} = 2s_{ij} -2\Psi \delta _{ij}$
\end{itemize}
where $\Psi $ encodes the trace of $h_{ij}$ and $s_{ij}$ is traceless.\par
Let's what can we say about $\Psi $. The trace of \emph{h} is 
\[
h \equiv \eta ^{\mu \nu } h_{\mu \nu }
\]
so 
\begin{equation}
	\delta _{ij}h_{ij} = 2\left( \delta _{ij}s_{ij} \right)-2\Psi \left( \delta _{ij}\delta _{ij} \right) = -6\Psi \to \Psi = -\frac{1}{6}\delta _{ij}h_{ij}
\end{equation}
in this case we don't care about the position of the indices because in Minkowski spatial the metric is the identity matrix; you can see that in the second step the first term just disappear because the $s$ is traceless.\par
In the context of \textbf{weak field}, \textbf{staticness} and \textbf{slow motion} the metric is
\[
ds^{2} = - \left( 1+2\phi  \right)dt^{2}+dx^{2}+dy^{2}+dz^{2} = -\left( 1+2\phi  \right)dt^{2}+ \delta _{ij}dx^{i}dx_{j}		
\]
while in the case of just \textbf{weak field}, like ours
\[
ds^{2} = - \left( 1+2\phi  \right)dt^{2} + w_{i}\left( dtdx^{i}+dx^{1}dt \right)+\left[ 2s_{ij}+\left( 1-2\phi  \right)\delta _{ij} \right]dx^{i}dx^{j}
\]
This definitions of the parts of the perturbation tensor are not gauge or solving equations, we just defined some convenient notation. The traceless tensor $s_{ij}$ is known as \emph{strain}, and as we'll see later, it contains gravitational radiation. \par
$\Psi $ is invariant under rotations because trace of a tensor is invariant under rotations.\par
How do the components of the metric tensor evolve? According to the Einstein Equations\footnote{The Einstein equations are not just one, we write one in tensorial form, but they are many}, in the linear regime, obviously. We can say that $\phi , \Psi , w_{i}$ are not propagating degrees of freedom, while $s_{ij}$ is propagating

What does it mean? If you write the Einstein Equations, with the expressions for $R_{\mu \nu }$ and \emph{R}, and instead of $h$ you write its decomposition, you can find the left-hand side of the EEs in function of $\phi ,\Psi , w_{i},s_{ij}$. The solution you find fo r $\phi ,\Psi ,w_{i}$ only involve derivatives of space, so they do not evolve with time and it means that if I know $T_{\mu \nu }$ and \emph{s} I can infer the functional dependence of the others. The actual field that propagates is the field of the strain.























