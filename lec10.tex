\section{Lec 10}
\subsection{Differential form}
A \emph{p-form} or \emph{differential form} is a (0,p) that is completely antisymmetric.
Examples of p-forms
\begin{itemize}
	\item scalars are 0-forms
	\item dual vectors are 1-forms
	\item $\tilde{\epsilon }$ is a 4-form
\end{itemize}
P-forms do have a operation called \emph{wedge product}:\par
be A a p-form and B a q-form, then $\left( A \wedge B \right)$ is a (p+q)-form, in detail
\[
	\left( A \wedge B \right)_{\mu _{1}\ldots \mu _{p+q}} = \frac{\left( p+q \right)!}{p!q!} A_{[\mu _{1}\ldots }B_{\mu _{p+1}\ldots \mu _{p+q}]}
\]
So, for example, if A and B are both a 1-form,
\[
	\left( A \wedge B \right) = \frac{2!}{1!1!} A_{[\mu }B_{\nu ]} = \frac{2!}{1!} \frac{1}{2!} \left( A_{\mu }B_{\nu }-A_{\nu }B_{\mu } \right) = A_{\mu }B_{\nu }-A_{\nu }B_{\mu }	
\]
Also note that
\[
A\wedge B = \left( -1 \right)^{pq} B\wedge A	
\]
There is also something about Integral over volume \& principle of least action. To Be Filled

\subsubsection{Exterior Derivative}
There is also this operation, not really used, that is
\[
d : p-form \to \left( p+1 \right)-form
\]
\[
	\left( dA \right)_{\mu _{1}\ldots \mu _{p+1} } = \left( p+1 \right) \partial_{[\mu _{1}}A_{\mu _{2}\ldots \mu _{n}]}
\]
It has a special property: \emph{dA} is a tensor.
\begin{gather}
\partial_{\alpha }A_{\beta  \gamma  \delta \ldots } \text{ is not a tensor, we already saw that, because of the extra piece that is symmetric and become 0 by anti-symmetrization }\\
\partial_{[\alpha }A_{\beta  \gamma  \delta  \ldots ]} \text{ is a tensor! }
\end{gather}
Now let's see how this is related to integrals.\par
Be in 3D, can be cartesian coordinates (x,y,z) or spherical (r, $\theta $, $\phi $), and be the gravitational field $\Phi\left( x,y,z \right)$ or $\Phi \left( r,\theta ,\phi  \right)$. What's the integral over space of $\Phi $?
\[
\int_{space}^{}{\Phi dV}
\]
with 
\[
dV = dxdydz = \left( r^{2}sin^{2}\theta  \right)drd\theta d\phi 
\]
Thinking about our guidelines: we want to describe independently on the chosen coordinates. $\Phi $ is a scalar, so let's see it's integration
\[
I = \int_{}^{}{\Phi \left( x \right) d^{n}x} \neq \int_{}^{}{\phi \left( x' \right)d^{n}x'}
\]
because
\[
d^{n}x^{\mu '} = \left| \frac{\partial x^{\mu '}}{\partial x^{\mu }}\right| d^{n}x^{\mu }
\]
there is a Jacobian in there.\par
The integrand of an integral is a p-form, and the integral a real number.
\[
d^{n}x = dx^{0}\wedge \ldots  \wedge dx^{n-1} = \frac{1}{n!} \tilde{\epsilon }_{\mu_{1}\ldots \mu _{n}}dx^{\mu _{1}} \wedge \ldots \wedge dx^{\mu _{n}}
\]
This is the integration measure. When I change coordinate system I get
\begin{gather}
d^{n}x = \frac{1}{n!} \tilde{\epsilon}_{\mu_{1}\ldots \mu _{n}} \left( dx^{\mu _{1}} \wedge \ldots dx^{\mu _{n}} \right) = \frac{1}{n!}\tilde{\epsilon }_{\mu _{1}\ldots \mu _{n}} \frac{\partial x^{\mu _{1}}}{\partial x^{\mu _{1}'}} \ldots \frac{\partial x^{\mu _{n}}}{\partial x^{\mu _{n}'}} \times \left( dx^{\mu _{1}'} \wedge \ldots \wedge dx^{\mu _{n}'} \right) = \\
= \frac{1}{n!} \tilde{\epsilon }_{\mu _{1}' \ldots \mu_{n}'} det\left( \frac{\partial x^{\mu }}{\partial x^{\mu }'}  \right)\left( dx^{\mu_{1}}\ldots dx^{\mu _{n}} \right)
\end{gather}
What did I get? That 
\[
	d^{n}x = \left[ det\left( \frac{\partial x^{\mu }'}{\partial x^{\mu }}  \right)\right]^{-1} d^{n}x'
\]
$d^{n}x$ is not a tensor but it is a tensor density.
We want an invariant integration measure $\sqrt{|g|}d^{n}x$, such that if $\Phi $ is a scalar also
\[
	\int_{}^{}{d^{n}x \sqrt{|g|}\Phi }
\]
is a scalar.\par

Now why $\partial_{\alpha }$ does not give a tensor? $\partial_{\mu }A_{\nu }$ is not a tensor.
\[
	\partial_{\mu '}A_{\nu '} = \frac{\partial x^{\mu }}{\partial x^{\mu '}} \frac{\partial x^{\nu }}{\partial x^{\nu '}} \partial_{\mu } A_{\nu } + \frac{\partial^{2}x^{\nu }}{\partial x^{\mu '}x^{\nu '}} A_{\nu }
\]

