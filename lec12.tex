\section{Lec 12}
\subsection{Geodesic Equation}
Given a generic path (WL) $x^{\mu }\left( \lambda  \right)$, parallel transport of a generic tensor is
\[
\frac{D}{d\lambda } \left( T^{\mu _{1}\ldots \mu _{k}} \right)_{\nu _{1}\ldots \nu _{l}} =\frac{d x^{\sigma }}{d \lambda } \nabla _{\sigma } T^{\mu _{1}\ldots \mu _{k}}_{\nu _{1}\ldots \nu _{l}} = 0		
\]
while for a tensor is 
\[
	\frac{D}{d\lambda } \left( V^{\rho } \right) = \frac{d x^{\mu }}{d \lambda } \nabla _{\mu }V^{\rho } = \frac{d x^{\mu }}{d \lambda } \left[ \partial_{\mu }V^{\rho } + \Gamma ^{\rho }_{\mu \sigma  } V^{\sigma } \right]=0
\]
We can look at the parallel transport equations as a first-order differential equation defining an initial-value problem: given a tensor at some point along the path, there will be a unique continuation of the tensor to other points along the path.

Probably was not said before but to make the parallel transport properly tensorial we need to replace the partial derivative by a covariant one, and define the \emph{directional covariant derivative} as
\[
\frac{D}{d\lambda } = \frac{d x^{\mu }}{d \lambda }\nabla _{\mu }
\]

We will review \emph{geodesic equation} and derive it again with another method.

\subsubsection{Geodesic I}
Geodesic is a path that transport its own tangent vector. That's the definition we will use for the first method. We schematized this in Lec 11, fig. \ref{imm:straightline}.
\begin{gather*}
\frac{D}{d\lambda } \left( \frac{d x^{\mu }}{d \lambda } \right) = 0 \\
\to  \frac{d x^{\sigma }}{d \lambda }\nabla _{\sigma }\left( \frac{d x^{\mu }}{d \lambda } \right) = \frac{d x^{\sigma }}{d \lambda }\left[ \partial_{\sigma } \left( \frac{d x^{\mu }}{d \lambda } \right) + \Gamma ^{\mu }_{\sigma \rho } \frac{d x^{\rho }}{d \lambda } \right] = 0 \\
\to \frac{d ^{2} x^{\mu }}{d \lambda ^{2}} + \Gamma ^{\mu }_{\rho \sigma } \frac{d x^{\rho }}{d \lambda } \frac{d x^{\sigma }}{d \lambda } 
\end{gather*}
so we get the \emph{geodesic equation.}

\subsubsection{Geodesic II}

This time let's start from the concept of distance so
\[
ds^{2} = g_{\mu \nu } dx^{\mu }dx^{\nu }
\]
There is no point in trying to minimize it, because if it's the path of a photon it is 0.
