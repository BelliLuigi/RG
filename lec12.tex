\section{Lec 12}
\subsection{Geodesic Equation}
Given a generic path (WL) $x^{\mu }\left( \lambda  \right)$, parallel transport of a generic tensor is
\[
\frac{D}{d\lambda } \left( T^{\mu _{1}\ldots \mu _{k}}_{\nu _{1}\ldots \nu _{l}}\right) =\frac{d x^{\sigma }}{d \lambda } \nabla _{\sigma } T^{\mu _{1}\ldots \mu _{k}}_{\nu _{1}\ldots \nu _{l}} = 0		
\]
while for a vector is 
\[
	\frac{D}{d\lambda } \left( V^{\rho } \right) = \frac{d x^{\mu }}{d \lambda } \nabla _{\mu }V^{\rho } = \frac{d x^{\mu }}{d \lambda } \left[ \partial_{\mu }V^{\rho } + \Gamma ^{\rho }_{\mu \sigma  } V^{\sigma } \right]=0
\]
We can look at the parallel transport equations as a first-order differential equation defining an initial-value problem: given a tensor at some point along the path, there will be a unique continuation of the tensor to other points along the path.

Probably it was not said before but to make the parallel transport properly tensorial we need to replace the partial derivative by a covariant one, and define the \emph{directional covariant derivative} as
\[
\frac{D}{d\lambda } = \frac{d x^{\mu }}{d \lambda }\nabla _{\mu }
\]

We will review \emph{geodesic equation} and derive it again with a different method.

\subsubsection{Geodesic I}
Geodesic is a path that transport its own tangent vector. That's the definition we will use for the first method. We schematized this in Lec 11, fig. \ref{imm:straigthline}.

\begin{gather*}
\frac{D}{d\lambda } \left( \frac{d x^{\mu }}{d \lambda } \right) = 0 \\
\to  \frac{d x^{\sigma }}{d \lambda }\nabla _{\sigma }\left( \frac{d x^{\mu }}{d \lambda } \right) = \frac{d x^{\sigma }}{d \lambda }\left[ \partial_{\sigma } \left( \frac{d x^{\mu }}{d \lambda } \right) + \Gamma ^{\mu }_{\sigma \rho } \frac{d x^{\rho }}{d \lambda } \right] = 0 \\
\to \frac{d ^{2} x^{\mu }}{d \lambda ^{2}} + \Gamma ^{\mu }_{\rho \sigma } \frac{d x^{\rho }}{d \lambda } \frac{d x^{\sigma }}{d \lambda } = 0 
\end{gather*}
so we get the \emph{geodesic equation.}

\subsubsection{Geodesic II}

This time let's start from the concept of distance so
\[
ds^{2} = g_{\mu \nu } dx^{\mu }dx^{\nu }
\]
There is no point in trying to minimize it, because if it's the path of a photon it is 0, it's already minimized for each null-path.\par
Since the metric is compatible with the connection we are using, we take the parallel transportation of this quantity, that is a scalar, $g_{\mu \nu }V^{\mu }W^{\nu }$:
\begin{equation}
\frac{D}{d\lambda } \left( g_{\mu \nu }V^{\mu }W^{\nu }	 \right) = \left( \frac{D}{d\lambda } g_{\mu \nu } \right)V^{\mu }W^{\nu } + g_{\mu \nu } \frac{D}{d\lambda }\left( V^{\mu }\right)W^{\nu }+ g_{\mu \nu} V^{\mu } \frac{D }{d \lambda } \left( W^{\nu } \right)
\end{equation}
we see that the first term on the right is 0 because of \emph{metric compatibility} and the other two are 0 as well because we are parallel transporting a vector\footnote{and the vector does not change}.\par
What does it mean? It means that scalar product is preserved by parallel transportation.\par
We will keep this lemma in mind while deriving the Geodesics Equation.

Now since in Lorentzian spacetime the definition of distance is kinda tricky let's stick to use proper time $\tau $, instead. So for a time-like trajectory, like the one of a massive particle, we have
\[
d\tau ^{2}= -g_{\mu \nu }dx^{\mu }dx^{\nu } \text{ and } d\tau ^{2}> 0
\]
For every trajectory we fix the extremes so we get the so-called \emph{proper time functional}
\[
\tau \equiv \int_{path}^{}\sqrt{-g_{\mu \nu } \frac{d x^{\mu }}{d \lambda } \frac{d x^{\nu }}{d \lambda }} d\lambda 	
\]
To search for the shortest distance one could do a thing called \emph{calculus of variations} or alternatively, recalling the twin paradox, recognize that the twin that experience more time is the one who stands still on the Earth. But we will try to check this out anyway.  We can simplify the algebra writing the integral above as
\begin{gather*}
	\tau  \equiv \int_{path}^{}{\sqrt{-f} d\lambda } \\
	\text{ with } f = g_{\mu \nu } \frac{d x^{\mu }}{d \lambda } \frac{d x^{\nu }}{d \lambda }.
\end{gather*}
Then we add perturbations to the path, and we pick the one with the maximum proper time.

To extremize the proper time functional, we need that its variation is null. This implies that the tangent vector to the path $\frac{d x^{\mu }}{d \lambda }$, behaves consistently along the curve. This vector is normalized such that the scalar product \emph{f} is constant, and the fact that parallel transport preserves the scalar product guarantees that this holds along the path.\par
i%So, extremes called A and B are fixed, we took a generic trajectory $x^{\mu }\left( \lambda  \right)$.
We consider a \emph{perturbation} of this trajectory such that
\begin{gather*}
x^{\mu  }\to x^{\mu }\delta x^{\mu } \\
g_{\mu \nu } \to  g_{\mu \nu } + \partial_{\rho }g_{\mu \nu } \delta  x^{\rho }
\end{gather*}
Where the second line comes from Taylor expansion in CST, which uses partial derivative not covariant one, because we are thinking of the components of $g_{\mu \nu }$ as functions of spacetime in some specific coordinates system.
Since we modify the trajectory, $\tau$ also changes
\[
	\delta \tau = \int_{}^{}{\delta \sqrt{-f}d\lambda } = \int_{}^{}{-\frac{1}{2\sqrt{-f}}\delta f d\lambda }
\]
Now we switch parameter to $\tau $ itself, this makes the four-velocity to become the tangent vector, and consequently the value of \emph{f} is now fixed at 
\[
f = g_{\mu \nu }\frac{d x^{\mu }}{d \tau }\frac{d x^{\nu }}{d \tau } = g_{\mu \nu }U^{\mu }U^{\nu } = -1
\]
and so
\[
\delta \tau = -\frac{1}{2} \int_{}^{}{\delta f d\tau }
\]
Stationary points of the \emph{functional of the proper time}, ( path for which $\delta  \tau =0$), are equivalent to stationary points of this simpler integral
\[
I = \frac{1}{2} \int_{}^{}{f d\tau } = \frac{1}{2} \int_{}^{}{g_{\mu \nu }\frac{d x^{\mu }}{d \lambda }\frac{d x^{\nu }}{d \lambda }}
\]
adding the perturbations on \emph{I} gets us
\begin{gather}\label{eq:deltaI}
\delta I = \frac{1}{2} \int_{}^{}{\left( \partial_{\sigma }g_{\mu \nu }\frac{d x^{\mu }}{d \tau }\frac{d x^{\nu }}{d \tau }\delta x^{\sigma } + g_{\mu \nu } \frac{d \left( \delta x^{\mu } \right)}{d \tau }\frac{d x^{\nu }}{d \tau } + g_{\mu \nu }\frac{d x_{\mu }}{d \tau } \frac{d \left(\delta  x^{\nu } \right)}{d \tau } \right)d\tau } \\
= \frac{1}{2} \int_{}^{}{\left( \partial_{\rho }g_{\mu \nu }\delta x^{\rho } \frac{d x^{\mu }}{d \tau } \frac{d x^{\nu }}{d \tau } +  g_{\mu \nu }\frac{d }{d \tau } \delta \left( x^{\mu } \right) \frac{d x^{\nu }}{d \tau } + g_{\mu \nu } \frac{d x^{\mu }}{d \tau } \frac{d }{d \tau } \delta \left( x^{\nu } \right) \right) d\tau }
\end{gather}
The last two terms can be integrated by parts,
\begin{gather*}
	\int_{}^{}{g_{\mu \nu } \frac{d }{d \tau }\left( \delta x^{\mu } \right) \frac{d x^{\nu }}{d \tau } d\tau } = \text{ boundary term } - \int_{}^{}{\frac{d }{d \tau } \left( g_{\mu \nu }\frac{d x^{\nu }}{d \tau } \right) \delta x^{\mu } d\tau } \\
 = - \int_{}^{}{\left( \partial_{\rho }g_{\mu \nu } \frac{d x^{\rho }}{d \tau } \frac{d x^{\nu }}{d \tau } + g_{\mu \nu } \frac{d ^{2} x^{\nu }}{d \tau ^{2}} \right) \delta x^{\mu } d\tau }
\end{gather*}
the boundary term vanishes because we take our variation $\delta x^{\mu }$ to vanish at the endpoints of the path. \par
So now we can insert back what we found in eq. \ref{eq:deltaI}
\begin{gather*}
\delta I = \frac{1}{2} \int_{}^{}{\partial_{\rho }g_{\mu \nu } \frac{d x^{\nu }}{d \tau } \frac{d x^{\mu  }}{d \tau } \delta x^{\rho } d\tau } - \frac{1}{2} \int_{}^{}{\left( \partial_{\rho } g_{\mu \nu } \frac{d x^{\rho }}{d \tau } \frac{d x^{\nu }}{d \tau } + g_{\mu \nu } \frac{d ^{2}x^{\nu }}{d \tau ^{2}} \right) \delta x^{\mu }d\tau }  +\\
- \frac{1}{2} \int_{}^{}{\left( \partial_{\rho } g_{\mu \nu } \frac{d x^{\rho }}{d \tau } \frac{d x^{\mu }}{d \tau } + g_{\mu \nu } \frac{d ^{2} x^{\mu }}{d \tau ^{2}} \right) \delta x^{\nu } d\tau }
\end{gather*}
We can make things easier if we use:
\begin{itemize}
\item $\alpha $ instead of $\rho $ in the first term
\item $\alpha $ instead of $\mu $ in the second term
\item $\alpha $ instead of $\nu $ in the third term
\item $\rho $ in second term is now $\mu $
\item $\rho $ in third term is now $\nu $
\item $\nu $ in one derivative of four-velocity is now $\mu $
\end{itemize}
to sum up, you need index of $\delta x$ be $\alpha $, and $\mu ,\nu $ on four velocities, and we get
\begin{equation}
	\delta I = \int_{}^{}{ \left[ \frac{\left( \partial_{\alpha } g_{\mu \nu } - \partial_{\mu } g_{\alpha \nu } - \partial_{\nu }g_{\mu \alpha } \right)}{2} \frac{d x^{\mu }}{d \tau } \frac{d x^{\nu }}{d \tau } - \frac{2}{2} g_{\mu \alpha } \frac{d ^{2}x^{\mu }}{d \tau ^{2}} \right] \delta x^{\alpha }d\tau }
\end{equation}
%{\footnotesize this has to be checked doing math by hand because probably there is a minus sign outside}\par DONE IT IS OK
Since we are searching for stationary points, we want $\delta I$ to vanish for any variation $\delta x^{\alpha } $, this implies 
\begin{equation}
g_{\beta \alpha } \frac{d ^{2}x^{\beta }}{d \tau ^{2}} + \frac{1}{2} \left( \partial_{\mu }g_{\alpha \nu } + \partial_{\nu } g_{\alpha \mu } - \partial_{\alpha }g_{\mu \nu } \right) \frac{d x^{\mu }}{d \tau } \frac{d x^{\nu }}{d \tau } = 0
\end{equation}
now if I multiply for the inverse metric $g^{\rho \alpha }$ the full expression
\begin{equation}
	\frac{d ^{2}x^{\rho }}{d \tau ^{2}} + \frac{g^{\rho \alpha }}{2} \left[ \partial_{\mu }g_{\alpha \nu } + \partial_{\nu } g_{\mu \alpha } - \partial_{\alpha } g_{\mu \nu }\right] \frac{d x^{\mu }}{d \tau }\frac{d x^{\nu }}{d \tau } = 0
\end{equation}
We see that we got the \emph{Geodesic Equation} but with a specific choice for the Christoffel connection.\par

\paragraph{Some comments on geodesic}
I can change variables, like $\tau \to \alpha \lambda + \beta $ but the geodesic equation does not change.\par

The first derivation was hiding that $ \lambda $ need to be related to $\tau $ in a linear way. Since
\begin{equation}
u^{\mu }\nabla _{\mu }u^{\nu } = 0 \text{ or  } p^{\mu }\nabla _{\mu }p^{\nu } = 0
\end{equation}
geodesics are trajectories defined by this equation, freely falling particles move like this.\footnote{Idk wtf i'm talking about here. Check urself or skip}

































