\section{2\textsuperscript{nd} question}

\begin{itemize}
	\item Write the definition of a covariant derivative. Describe how the connection coefficients change under a coordinate change. Write a demonstration on how to get the covariant derivative of a dual vector. (See \ref{sec:covder})
	\item calculate the commutator of 2 covariant derivatives acting on a vector V, and to indicate in the end the riemann tensor and the component given by the torsion (See \ref{sec:Rtns})
	\item Derive the non-vanishing Christoffel symbols of the RW metric. (See Lec 23)
	\item Derive the gravitational redshift as a consequence of the Schwarzschild metric. (See \ref{sec:grSred})
	\item Derive the energy density relations using the conservation equations of a perfect fluid and the RW metric. (See \ref{sec:energydensityrelRW} )
\item Show that a general spherically symmetric metric depends only on two functions $\alpha \left( r \right)$ and $\beta \left( r \right)$ then obtain the non-vanishing Christoffel symbols from the resulting relation.(See Lec16)
\item Derive the conserved energy-momentum tensor for a perfect fluid then using this tensor derive the continuity equation. (See Lec24)
\item Show why $R_{\mu \nu } = k T_{\mu \nu }$ is a problematic guess to the energy-momentum relation, instead derive a good value of k that satisfies this relation. (See Lec 15)
\item Given the conditions for the Newtonian limit, show that $g_{00 } = - \left( 1-2\phi  \right) $ (First part of Lec 15)
\item Given the wave function $\Psi $ and the properties of $h^{TT}_{\mu \nu }$, derive the amplitude tensor $C_{\mu \nu }$ in terms of $h_{+}$ and $h_{\times}$.(See Lec22)
\item Derive the conservation of energy in Schwartzschild metric and discuss the stability of every possible circular orbit. (See Lec17 section \emph{Planets})
\item Calculate the independent components of the Riemann tensor. Then calculate for the specific case n=4 (See Lec 14)
\item Show why the partial derivative of a vector $\partial_{\mu }V^{\mu }$ does not transform as a tensor. (See Lec8 Section \emph{Special Tensors})
\item Show that the torsion is a tensor then considering a torsion-free condition, derive the Christoffel symbol.(See Lec11 sec \emph{Torsion} + \emph{Metric Compatibility})
\item Using the stress-energy tensor for a perfect fluid with the cosmological constant in RW, derive the energy density and the pressure. (See lec24)
\item Describe the deflection of light as it passes a massive object, like the sun, using linearized gravity. (See sec 4.2.1)
\item Derive the geodesic equation by maximizing the proper time. (See Lec12)
\item Derive the Christoffel symbols, Riemann Tensor and Ricci Tensor given $g_{\mu \nu } = \eta_{\mu \nu } +h_{\mu \nu }$. 
\end{itemize}












