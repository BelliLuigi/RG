\section{Lec 15}

\subsection{Einstein Equation}
Today we will derive the Einstein Equation. We will use a technique called \emph{minimal coupling principle}. It has three steps
\begin{enumerate}
\item Take a physics law valid in flat spacetime (e.g. poisson equation for gravitational potential).
\item write it in a coordinates independent form ( tensorial).
\item Assume it is valid for curved spacetime.
\end{enumerate}

Let's start with the first step. \par
We consider the motion of freely-falling particles. From cartesian coordinates, $x^{\mu }$, we have a straigt line \[
\frac{d ^{2}x^{\mu }}{d \lambda ^{2}} = 0 
\]
n.b. this is not a tensorial relation: $\frac{d x^{\mu }}{d \lambda }$ is a well-defined vector, but the second derivative is not.
Now I change coordinates, but keeping spacetime flat. This is because a law that changes with coordinates is not coordinates independent.
\[
\frac{d }{d \lambda }\left( \frac{d x^{\mu }}{d \lambda } \right) = 0
\]
this one is composed by a vector inside the brackets, while the operator outside is not a tensor. \par
We could use the chain rule to write
\[
\frac{d ^{2}x^{\mu }}{d \lambda ^{2}} = \frac{d x^{\sigma }}{d \lambda }\partial_{\sigma }\frac{d x^{\mu }}{d \lambda }
\]
but the partial derivative is still a problem. Maybe we should use the covariant one.
So we get
\[
	\frac{d x^{\sigma }}{d \lambda }\nabla _{\sigma } \left( \frac{d x^{\mu }}{d \lambda } \right) = \frac{d x^{\sigma }}{d \lambda }\left[ \partial_{\sigma } \left( \frac{d x^{\mu }}{d \lambda } \right) + \Gamma ^{\mu }_{\sigma \alpha } \frac{d x^{\alpha }}{d \lambda }\right] = 0
\]
in the end
\[
\frac{d ^{2}x^{\mu }}{d \lambda ^{2}} + \Gamma ^{\mu }_{\sigma \alpha }\frac{d x^{\sigma }}{d \lambda }\frac{d x^{\alpha }}{d \lambda } = 0
\]
This is the geodesic equation. In general relativity we said freely-falling particles move along geodesics. Now we have generalized a small thing to curved spacetime but it is different from saying that this describes gravity.\par
So, let's show how results from the Newtonian limit fit in this picture. This limit has three requirements
\begin{itemize}
\item slowly moving particles $v \ll c$
\item weak gravitational field, so the metric could be Minkowskian with a little perturbation
	\[
		g_{\mu \nu } = \eta _{\mu \nu } + h_{\mu \nu }, |h_{\mu \nu }|\ll 1
	\]
\item the gravitational field is also static, it does not change in time ( $\partial_{0}g_{\mu \nu } = 0$)
\end{itemize}
We consider time-like geodesics, so trajectories of massive particles, so it is useful to parametrize them using the proper time $\tau $.
\[
\frac{d ^{2}x^{\mu }}{d \tau ^{2}} + \Gamma ^{\mu }_{\alpha  \beta } \frac{d x^{\alpha }}{d \tau }\frac{d x^{\beta }}{d \tau } = 0
\]
I want to use this to describe the motion under gravitational field, so we want to recover the $\vec{a} = -\vec{\nabla } \phi $. $\phi $ is hidden inside the connection, while $\vec{a}$ is the second derivative. \par
Since we required slow motion 
\[
\frac{d t}{d \tau } \gg \frac{d x^{i}}{d \tau }	
\]
So the geodesics equation becomes
\begin{equation}
\frac{d ^{2}x^{\mu }}{d \tau ^{2}} + \Gamma ^{\mu }_{00}\frac{d x^{0}}{d \tau }\frac{d x^{0}}{d \tau } = 0 \to  \frac{d ^{2}x^{\mu }}{d \tau ^{2}} + \Gamma ^{\mu }_{00}\left( \frac{d t}{d \tau } \right)^{2} = 0
\end{equation}
In this situation I just need 4 entries for the connection, that is defined as
\[
	\Gamma ^{\rho }_{\mu \nu } = \frac{1}{2} g^{\rho \sigma }\left[ \partial_{\mu } g_{\sigma\nu } + \partial_{\nu }g_{\sigma \mu } - \partial_{\sigma }g_{\mu \nu } \right]
\]
so
\begin{equation}
	\Gamma ^{\mu }_{00} = \frac{1}{2} g^{\mu \alpha } \left[ \partial_{0}g_{\alpha 0} + \partial_{0}g_{\alpha 0} - \partial_{\alpha }g_{00} \right]
\end{equation}
since the condition of a static gravitational field, the time derivatives are equal to zero. We are left with 
\[
\partial_{\alpha }g_{00} = \partial_{\alpha }\left( \eta _{00} + h_{00} \right) = \partial_{\alpha }h_{00}
\]
because the Minkowskian metric tensor is constant. 
\begin{equation}
\Gamma ^{\mu }_{00} = -\frac{1}{2} g^{\mu \alpha } \partial_{\alpha }h_{00} = 
\end{equation}






















