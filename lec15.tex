\section{Lec 15}

\subsection{Einstein Equation}
Today we will derive the Einstein Equation. We will use a technique called \emph{minimal coupling principle}. It has three steps
\begin{enumerate}
\item Take a physics law valid in flat spacetime (e.g. poisson equation for gravitational potential).
\item write it in a coordinates independent form ( tensorial).
\item Assume it is valid for curved spacetime.
\end{enumerate}

Let's start with the first step. \par
We consider the motion of freely-falling particles. From cartesian coordinates, $x^{\mu }$, we have a straigt line \[
\frac{d ^{2}x^{\mu }}{d \lambda ^{2}} = 0 
\]
n.b. this is not a tensorial relation: $\frac{d x^{\mu }}{d \lambda }$ is a well-defined vector, but the second derivative is not.
Now I change coordinates, but keeping spacetime flat. This is because a law that changes with coordinates is not coordinates independent.
\[
\frac{d }{d \lambda }\left( \frac{d x^{\mu }}{d \lambda } \right) = 0
\]
this one is composed by a vector inside the brackets, while the operator outside is not a tensor. \par
We could use the chain rule to write
\[
\frac{d ^{2}x^{\mu }}{d \lambda ^{2}} = \frac{d x^{\sigma }}{d \lambda }\partial_{\sigma }\frac{d x^{\mu }}{d \lambda }
\]
but the partial derivative is still a problem. Maybe we should use the covariant one.
So we get
\[
	\frac{d x^{\sigma }}{d \lambda }\nabla _{\sigma } \left( \frac{d x^{\mu }}{d \lambda } \right) = \frac{d x^{\sigma }}{d \lambda }\left[ \partial_{\sigma } \left( \frac{d x^{\mu }}{d \lambda } \right) + \Gamma ^{\mu }_{\sigma \alpha } \frac{d x^{\alpha }}{d \lambda }\right] = 0
\]
in the end
\[
\frac{d ^{2}x^{\mu }}{d \lambda ^{2}} + \Gamma ^{\mu }_{\sigma \alpha }\frac{d x^{\sigma }}{d \lambda }\frac{d x^{\alpha }}{d \lambda } = 0
\]
This is the geodesic equation. In general relativity we said freely-falling particles move along geodesics. Now we have generalized a small thing to curved spacetime but it is different from saying that this describes gravity.\par
So, let's show how results from the Newtonian limit fit in this picture. This limit has three requirements
\begin{itemize}
\item slowly moving particles $v \ll c$
\item weak gravitational field, so the metric could be Minkowskian with a little perturbation
	\[
		g_{\mu \nu } = \eta _{\mu \nu } + h_{\mu \nu }, |h_{\mu \nu }|\ll 1
	\]
\item the gravitational field is also static, it does not change in time ( $\partial_{0}g_{\mu \nu } = 0$)
\end{itemize}
We consider time-like geodesics, so trajectories of massive particles, so it is useful to parametrize them using the proper time $\tau $.
\[
\frac{d ^{2}x^{\mu }}{d \tau ^{2}} + \Gamma ^{\mu }_{\alpha  \beta } \frac{d x^{\alpha }}{d \tau }\frac{d x^{\beta }}{d \tau } = 0
\]
I want to use this to describe the motion under gravitational field, so we want to recover the $\vec{a} = -\vec{\nabla } \phi $. $\phi $ is hidden inside the connection, while $\vec{a}$ is the second derivative. \par
Since we required slow motion 
\[
\frac{d t}{d \tau } \gg \frac{d x^{i}}{d \tau }	
\]
So the geodesics equation becomes
\begin{equation}
\frac{d ^{2}x^{\mu }}{d \tau ^{2}} + \Gamma ^{\mu }_{00}\frac{d x^{0}}{d \tau }\frac{d x^{0}}{d \tau } = 0 \to  \frac{d ^{2}x^{\mu }}{d \tau ^{2}} + \Gamma ^{\mu }_{00}\left( \frac{d t}{d \tau } \right)^{2} = 0
\end{equation}
In this situation I just need 4 entries for the connection, that is defined as
\[
	\Gamma ^{\rho }_{\mu \nu } = \frac{1}{2} g^{\rho \sigma }\left[ \partial_{\mu } g_{\sigma\nu } + \partial_{\nu }g_{\sigma \mu } - \partial_{\sigma }g_{\mu \nu } \right]
\]
so
\begin{equation}
	\Gamma ^{\mu }_{00} = \frac{1}{2} g^{\mu \alpha } \left[ \partial_{0}g_{\alpha 0} + \partial_{0}g_{\alpha 0} - \partial_{\alpha }g_{00} \right]
\end{equation}
since the condition of a static gravitational field, the time derivatives are equal to zero. We are left with 
\[
\partial_{\alpha }g_{00} = \partial_{\alpha }\left( \eta _{00} + h_{00} \right) = \partial_{\alpha }h_{00}
\]
because the Minkowskian metric tensor is constant.
Now, since the metric tensor is the Minkowskian one plus a little perturbation it could be
assumed directly as minkowskian, so it is a first-order approximation.
\begin{equation}
\Gamma ^{\mu }_{00} = -\frac{1}{2} g^{\mu \alpha } \partial_{\alpha }h_{00} = - \frac{1}{2}\eta ^{\mu \alpha } \partial_{\alpha }h_{00} = - \frac{1}{2} d^{\mu }h_{00}	
\end{equation}

At this point we can start to put together again the geodesic equation.
We have 
\begin{equation}
\frac{d ^{2}x^{\mu }}{d \tau ^{2}} = \frac{1}{2} \partial^{\mu }h_{00} \left( \frac{d t}{d \tau } \right)^{2}
\end{equation}
Now let's look what happens base on which coordinates we choose.\par
For $\mu  = 0$ we are left with
\[
\frac{d^{2}t }{d \tau ^{2} } = 0 \implies \frac{d t}{d \tau } = \text{ const }
\]
while for $\mu  = i$
\begin{equation}
\frac{d ^{2}x^{i}}{d \tau ^{2}} = \frac{1}{2} \left( \frac{d t}{d \tau } \right)^{2}\partial^{i}h_{00}\left( \frac{d t}{d \tau } \right)^{2} \to  \frac{d ^{2}x^{i}}{d t^{2} } = \frac{1}{2}\partial_{i}h_{00}
\end{equation}
this because we divided both sides by $\left( \frac{d t}{d \tau }\right)^{2}$.
As we said before the left term is exactly the acceleration
\[
a^{i} = \frac{1}{2} \partial^{i}h_{00} 
\]
and if $h_{00} = -2\Phi $ we get
\begin{equation}
	\vec{a} = -\vec{\nabla } \Phi 
\end{equation}
Also
\[
g_{00} = -\left( 1+ 2\Phi  \right)
\].
Now, we would like to substitute the poissonian equation for Newtonian potential.
\[
\nabla^{2} \Phi = 4\pi G\rho 	
\]

where $\nabla ^{2} = \delta ^{ij}\partial_{i}\partial_{j}  = \partial^{2}_{x}+\partial^{2}_{y}+\partial^{2}_{z}$. 

So we have a second-order differential operator  acting on the  gravitational potential and non the right side a measure of the mass distribution, but we want an equation between tensors. The tensor generalization of the mass density is the energy-momentum tensor $T_{\mu \nu }$. The gravitational potential should get replaced by the metric tensor, because as above we need perturbed the metric to reproduce gravity. So the first attempt could be like
\[
\nabla ^{\sigma }\nabla _{\sigma }\left( g_{\mu \nu } \right) = k T_{\mu \nu }
\]
but it won't work because of metric compatibility $\nabla _{\sigma }\left( g_{\mu \nu } \right) = 0$.\par

There is a quantity that is constructed from second derivative of the metric that is the Riemann tensor, but since it has too many indices we will resort to the Ricci tensor $R_{\mu \nu }$. So it could be something like
\[
R_{\mu \nu } = k T_{\mu \nu }
\]
with some constant \emph{k}. But there is a problem, with the conservation of energy. We like that 
\[
\nabla ^{\mu }T_{\mu \nu } = 0
\]
is true, this implies
\[
\nabla ^{\mu }R_{\mu \nu } = 0 \text{ \textbf{BUT} actually  } \nabla^{\mu }R_{\mu \nu } = \frac{1}{2}\nabla _{\nu }R
\]
That's an impasse. We will try to work it out writing the Bianchi Identity
\begin{gather*}
	g^{\nu \sigma }g^{\lambda \mu } \left[ \nabla _{\lambda }R_{\rho \sigma \nu \mu } + \nabla _{\rho }R_{\sigma \lambda \mu \nu } + \nabla _{\sigma }R_{\lambda \rho \mu \nu }\right] =0 \\
	\nabla ^{\mu }R_{\mu \rho }- \nabla _{\rho }R	+ \nabla ^{\nu }R_{\rho \nu } =0\\
	2\nabla ^{\mu }R_{\mu \rho } = \nabla _{\rho }R \to \nabla^{\mu }R_{\mu \rho } = \frac{1}{2} \nabla _{\rho }R \\
	\nabla ^{\mu }R_{\mu \rho } - \frac{1}{2} \nabla _{\rho }R = 0 \\
	\nabla ^{\mu }R_{\mu \rho } - \frac{1}{2} g_{\mu \rho } \nabla ^{\mu }R = 0 \\
	\nabla ^{\mu } \left( R_{\mu \rho } - \frac{1}{2} g_{\mu \rho }R \right) = 0
\end{gather*}
This is what we were looking for: a combination of Ricci tensor and scalar that has vanishing divergence!
\begin{equation}
R_{\mu \nu } - \frac{1}{2}g_{\mu \nu }R = k T_{\mu \nu }
\end{equation}















