\section{Lecture 4}

\subsubsection{Brief recap of lec3}
We defined vectors
\begin{itemize}
	\item localized at each spacetime point
	\item for each event P we defined the tangent space $T_{P}$
	\item there is linear combination inside $T_{P}$
	\item it has a basis
	\item Vectors and basis transform under LT Group.
\end{itemize}

\subsubsection{Dual vectors}
Using old terminology they are covariant, so with lower indices. Meanwhile contravariant do have upper indices.

Let's start with defining the \textbf{dual space} of a vector space:
\emph{Given a vector space (for concreteness T\textsubscript{P})}, we define the \textbf{dual space} $T_{P}^{*}$ as the space of linear maps between $T_{P}$ and $\mathbb{R}$.

\paragraph{Example}
Being $\omega  \in T_{P}^{*}$, $V \in T_{P}$ then
\[
	\omega \left( V \right) \in \mathbb{R}
\]
Linearity tells me that 
\[
\omega \left( \alpha V + \beta  W \right) = \alpha \omega \left( V \right) + \beta \omega \left( W \right)
\]

\paragraph{1\textsuperscript{st} statement}: The dual space is a vector space.
\[
	\left( \alpha \omega  + \beta \eta  \right)\left( v \right) = \alpha \omega \left( v \right) + \beta \eta \left( v \right)
\]
\paragraph{2\textsuperscript{nd} statement}: What is the dual of the dual?
\[
	\left( T_{P}^{*} \right)^{*} = T_{P} \implies v\left( \omega  \right) = \omega \left( v \right) \in \mathbb{R}
\]

\paragraph{Basis for T\textsubscript{P}\textsuperscript{*}}: $\hat{o}^{\left( \mu  \right)}$. \\
How to define this? Definition is
\[
\hat{o}^{\left( \mu  \right)}\left( \hat{e}_{\left( \nu  \right)} \right) \equiv \delta^{\mu }_{\nu }
\]
Now let's see if we can get how dual vectors work with vectors.
If I have:
\begin{itemize}
	\item generic item of $T_{P}$: $V = V^{\nu }\hat{e}_{\left( \nu  \right)}$
	\item generic item of $T_{P}^{*}: \omega = \omega_{\mu }\hat{o^{\left( \mu  \right)}}$ 
\end{itemize} 

I can compute:
\begin{gather*}
\omega \left( v \right) = \omega_{\mu }\hat{o}^{\left( \mu  \right)}\left( v^{\nu }\hat{e}_{\left( \nu  \right)} \right) =  \\
= \omega_{\mu }v^{\nu }\hat{o}^{\left( \mu  \right)}\left( \hat{e}_{\left( \nu  \right)} \right) = \omega_{\mu } v^{\nu } \delta^{\mu}_{\nu } = \omega_{\mu }v^{\mu }
\end{gather*}

\noindent\fbox{\begin{minipage}{\textwidth}
Once we know this we can do an \textbf{exercise}: show the way $\omega_{\mu '}$ transform.
What to do is to start from $\Lambda $ equality.\\
\end{minipage}} \\

What is the example of a dual vector? \\
\[
A_{\mu '} = \Lambda_{\mu }^{\mu '} A_{\mu }
\]
the gradient is a beautiful example of a \emph{dual vector}.
\[
A_{\mu } = \frac{\partial\phi }{\partial x^{\mu }} \text{  ;   } A_{\mu '} = \frac{\partial\phi }{\partial x^{\mu '}}
\]
This is useful to define LTs, in this way
\[
\frac{\partial\phi }{\partial x^{\mu '}} = \frac{\partial\phi }{\partial x^{\mu }} \frac{\partial x^{\mu }}{\partial x^{\mu '}} \to A_{\mu '} = \frac{\partial x^{\mu }}{\partial x^{\mu '}}A_{\mu }
\]
the LT is the last partial derivative. \\

There is a \emph{more compact} notation tho write partial derivatives that is
\[
\partial_{\mu } \phi \equiv \frac{\partial\phi }{\partial x^{\mu }}
\]

\subsection{Tensors}

Tensors are generalization of dual vectors and vectors. \\
They are \emph{multilinear maps}, i.e. functions of several variables and linear for all of them. For each tensor of \emph{rank} (k,l), we have
\[
T_{P}^{*}\times \ldots \times T_{P}^{*} \times T_{P}\times \ldots \times T_{P} \to \mathbb{R} \]
Where each dual vector space is present \textbf{k}-times, and vector space \textbf{l}-times.

Now let's see what is multilinearity on the combat field.

\noindent\fbox{
	\begin{minipage}{0.95\textwidth}
	Be a (1,1) tensor:
	\begin{itemize}
		\item $\alpha , \beta , \gamma , \delta \in \mathbb{R}$
		\item $\omega , \eta  \in T_{P}^{*}$
		\item v,w $\in $ T\textsubscript{P}
	\end{itemize}
	Given these we have
	\begin{equation}
	T\left( \alpha \omega + \beta \eta , \gamma v + \delta w \right) = \\
	= \alpha \gamma T\left( \omega , v \right) + \beta \delta T\left( \eta , w \right) + \alpha \delta T\left( \omega , w \right) + \beta \gamma T\left( \eta ,v \right)
	\end{equation}

\end{minipage}}

Once we have this general definition, let's take one step back:
\begin{itemize}
	\item Scalar $\to$ (0,0) tensor
	\item Vector $\to$ (1,0) tensor
	\item Dual vector $\to$ (0,1) tensor
\end{itemize}

\subsubsection{Tensor product}
Be:
\begin{itemize}
	\item T, rank (k,l) tensor
	\item S, rank (m,n) tensor
\end{itemize} 
We want to understand the action of $\otimes$. \\
So we know that $T\otimes S$ outputs (k+m, l+n) tensor. In particular,
\begin{gather*}
	T\otimes S \left[\omega^{\left( 1 \right)}, \ldots, \omega^{\left( k \right)}, \omega^{\left( k+1 \right)}, \ldots , \omega^{\left( k+m \right)}, v^{\left( 1 \right)}, \ldots , v^{\left( l \right)}, v^{\left( l+1 \right)}, \ldots , v^{\left( l+n \right)}\right] \equiv \\
	\equiv T\left( \omega^{\left( 1 \right)}, \ldots , \omega^{\left( k \right)}, v^{\left( 1 \right)}, \ldots , v^{\left( l \right)} \right) \times S\left( \omega^{\left( k+1 \right)}, \ldots , \omega^{\left( k+m \right)}, v^{\left( l+1 \right)}, \ldots , v^{\left( l+n \right)} \right) \\
	\implies T \otimes S \neq S \otimes T
\end{gather*}
so tensors do not commute.\\

\subsubsection{Basis for a tensor}
Let \emph{T} be a generic tensor with rank (k,l), \emph{basis} is given by
\[
\hat{e}_{\left( \mu_{1} \right)} \otimes \ldots \otimes \hat{e}_{( \mu_{k})} \otimes  \hat{o}^{\left( \nu_{1} \right)} \otimes \ldots \otimes \hat{o}^{(\nu_{l})}
\]

 A tensor can be written as 
 \[
 T = T^{\mu_{1}, \ldots , \mu_{k}}_{\nu_{1}, ..., \nu_{l}} \left( \hat{e}_{\left( \mu_{1} \right)} \otimes \ldots  \right) = T^{\mu_{1}^{'}, \ldots , \mu_{k}^{'}}_{\nu_{1}^{'}, \ldots , \nu _{l}^{'}}\left( \hat{e}_{\left( \mu_{1}^{'}  \right)} \otimes \ldots  \right)
 \]
So the tensor is always the same, the thing that changes is its components, because a change of RF I think.

We will often write the components instead of the actual tensor, but it is our convention to think they are equivalent.

This is how the components are related:
\begin{gather*}
\hat{e}_{\left( \mu ' \right)} = \Lambda^{\mu }_{\mu '} \hat{e}_{\left( \mu  \right)} \\
\hat{o}^{\left( \mu ' \right)} = \Lambda^{\mu '}_{\mu }\hat{o}^{\left( \mu  \right)} \\
\implies T = T^{\mu_{1}, \ldots , \mu_{k}}_{\nu_{1}, \ldots , \nu_{l}} \left( \Lambda^{\mu_{1}'}_{\mu_{1}} \hat{e}_{\left( \mu_{1}' \right)} \otimes \ldots  \right)
\end{gather*}
So we find, as result, that when I change frame
\begin{equation}
T^{\mu_{1}', \ldots , \mu_{k}'}_{\nu_{1}', \ldots , \nu_{k}'} = \Lambda^{\mu_{1}'}_{\mu_{1}} \ldots \Lambda^{\nu_{1}}_{\nu_{1}'} \ldots T^{\mu_{1}, ..., \mu_{k}}_{\nu_{1}, \ldots , \nu_{l}}
\end{equation}

