\chapter{Black Holes}
\section{Lec 16}
\subsection{Scharzschild's Metric}
The most obvious application of a theory of gravity is the case of spherical symmetry, that is like the one of the Earth or the Sun. We will start with a solution of the vacuum outside them, because is both easier and more useful. 
A specific solution of the Einstein Equation is the Schwarzschild Solution, for a static (components of $g_{\mu \nu }$ do not depend on time) spacetime with $\left( S^{2} \right)$ symmetry.\par
There is more than one way to derive this solution, we will use the most boring one:
\begin{enumerate}
\item \emph{Guess} a generic form for $g_{\mu \nu } $
\item Compute $\Gamma ^{\mu }_{\alpha \beta }$, $R^{\alpha }_{\beta \gamma \delta }$, $R_{\alpha \beta }$
\item Solve $R_{\mu \nu } = 0$, since the space outside the sun is empty space, vacuum.
\end{enumerate}

Let's start from the \textbf{guess} step.
We want to \emph{guess} the metric in spherical coordinates 
\begin{equation}\label{eq:polarmetric}
ds^{2}= -A\left( r \right) dt^{2} + B\left( r \right) dr^{2} + C\left( r \right) r^{2}d\Omega ^{2}
\end{equation}

where $d\Omega ^{2}$ is the metric on a $S^{2}$ sphere.
\[
	d\Omega ^{2} = d\theta ^{2} + sin^{2}\theta d\phi ^{2}
\]
Since we are almost in the vacuum, we can start in fact from the Minkowski metric in polar coordinates: $ds^{2 }=-dt^{2} +dr^{2}+r^{2}d\Omega ^{2} $.
Now we would like simplify a little eq.\ref{eq:polarmetric}:
\begin{itemize}
\item $r^{2}C\left( r \right) \to  r^{2}$, this is a change of coordinates, nothing to do with spherical symmetry, but know we have to rescale $A\left( r \right), B\left( r \right)$
\item $A \equiv e^{2\alpha \left( r \right)}$
\item $B\equiv e^{2\beta \left( r \right)}$
\end{itemize}.
With this, the most general guess is:
\begin{equation}\label{eq:pmetric2}
ds^{2} = -e^{2\alpha \left( r \right)}dt^{2} + e^{2\beta \left( r \right)}dr^{2} + r^{2}d\Omega ^{2}
\end{equation}
We have now two unknown functions $\alpha ,\beta $ of the radial coordinate, not time because of staticness, not angle because of isotropy.\par
Now it's time to \textbf{compute the metric tensor, Christoffel thing and Riemann tensor.}\par
\paragraph{Metric} 
\begin{equation}
\begin{matrix} 
g_{tt}= -e^{2\alpha \left( r \right)} & g_{rr}=e^{2\beta \left( r \right)} & g_{\theta \theta }= r^{2} & g_{\phi \phi } = r^{2}sin^{2}\theta  \\
g^{tt}=-e^{-2\alpha \left( r \right)} & g^{rr}=e^{-2\beta \left( r \right)} & g^{\theta \theta }= \frac{1}{r^{2}} & g^{\phi \phi }=\frac{1}{r^{2}sin^{2}\theta } \\
\end{matrix}
\end{equation}
other entries are null.
\paragraph{Christoffel Symbols}
 In most generic spacetimes there are 40 independent Christoffel coefficients. %unclear why there are less
As you may remember the Christoffel connection is defined as
\begin{equation}
	\Gamma ^{\rho }_{\mu \nu } = \frac{1}{2}g^{\rho \sigma }\left[ \partial_{\mu }g_{\sigma \nu }+\partial_{\nu }g_{\sigma \mu }-\partial_{\sigma }g_{\mu \nu }\right]
\end{equation}
if $\rho = t $:
\begin{equation}
	\Gamma^{t}_{\mu \nu } = \frac{1}{2} g^{t\sigma } \left[ \partial_{\mu }g_{\sigma \nu }+\partial_{\nu }g_{\sigma \mu }- \partial_{\sigma }g_{\mu \nu }\right]
\end{equation}
from this I get contribution only for $\sigma =t$, because of how is defined the metric. 
\begin{equation}
	\left( \sigma = t \right) \to \Gamma ^{t}_{\mu \nu } = \frac{1}{2}g^{tt}\left[ \partial_{\mu }g_{t\nu } + \partial_{\nu }g_{t\mu } - \partial_{t}g_{\mu \nu }\right]
\end{equation}
At this point, one immediately see that the last derivative is null since staticness.


























