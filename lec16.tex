\chapter{Black Holes}
\section{Lec 16}
\subsection{Scharzschild's Metric}
The most obvious application of a theory of gravity is the case of spherical symmetry, that is like the one of the Earth or the Sun. We will start with a solution of the vacuum outside them, because is both easier and more useful. 
A specific solution of the Einstein Equation is the Schwarzschild Solution, for a static (components of $g_{\mu \nu }$ do not depend on time) spacetime with $\left( S^{2} \right)$ symmetry.\par
There is more than one way to derive this solution, we will use the most boring one:
\begin{enumerate}
\item \emph{Guess} a generic form for $g_{\mu \nu } $
\item Compute $\Gamma ^{\mu }_{\alpha \beta }$, $R^{\alpha }_{\beta \gamma \delta }$, $R_{\alpha \beta }$
\item Solve $R_{\mu \nu } = 0$, since the space outside the sun is empty space, vacuum.
\end{enumerate}

Let's start from the \textbf{guess} step.
We want to \emph{guess} the metric in spherical coordinates 
\begin{equation}\label{eq:polarmetric}
ds^{2}= -A\left( r \right) dt^{2} + B\left( r \right) dr^{2} + C\left( r \right) r^{2}d\Omega ^{2}
\end{equation}

where $d\Omega ^{2}$ is the metric on a $S^{2}$ sphere.
\[
	d\Omega ^{2} = d\theta ^{2} + sin^{2}\theta d\phi ^{2}
\]
Since we are almost in the vacuum, we can start in fact from the Minkowski metric in polar coordinates: $ds^{2 }=-dt^{2} +dr^{2}+r^{2}d\Omega ^{2} $.
Now we would like simplify a little eq.\ref{eq:polarmetric}:
\begin{itemize}
\item $r^{2}C\left( r \right) \to  r^{2}$, this is a change of coordinates, nothing to do with spherical symmetry, but know we have to rescale $A\left( r \right), B\left( r \right)$
\item $A \equiv e^{2\alpha \left( r \right)}$
\item $B\equiv e^{2\beta \left( r \right)}$
\end{itemize}.
With this, the most general guess is:
\begin{equation}\label{eq:pmetric2}
ds^{2} = -e^{2\alpha \left( r \right)}dt^{2} + e^{2\beta \left( r \right)}dr^{2} + r^{2}d\Omega ^{2}
\end{equation}
We have now two unknown functions $\alpha ,\beta $ of the radial coordinate, not time because of staticness, not angle because of isotropy.\par
Now it's time to \textbf{compute the metric tensor, Christoffel thing and Riemann tensor.}\par
\paragraph{Metric} 
\begin{equation}
\begin{matrix} 
g_{tt}= -e^{2\alpha \left( r \right)} & g_{rr}=e^{2\beta \left( r \right)} & g_{\theta \theta }= r^{2} & g_{\phi \phi } = r^{2}sin^{2}\theta  \\
g^{tt}=-e^{-2\alpha \left( r \right)} & g^{rr}=e^{-2\beta \left( r \right)} & g^{\theta \theta }= \frac{1}{r^{2}} & g^{\phi \phi }=\frac{1}{r^{2}sin^{2}\theta } \\
\end{matrix}
\end{equation}
other entries are null.
\paragraph{Christoffel Symbols}
 In most generic spacetimes there are 40 independent Christoffel coefficients. %unclear why there are less
As you may remember the Christoffel connection is defined as
\begin{equation}
	\Gamma ^{\rho }_{\mu \nu } = \frac{1}{2}g^{\rho \sigma }\left[ \partial_{\mu }g_{\sigma \nu }+\partial_{\nu }g_{\sigma \mu }-\partial_{\sigma }g_{\mu \nu }\right]
\end{equation}
if $\rho = t $:
\begin{equation}
	\Gamma^{t}_{\mu \nu } = \frac{1}{2} g^{t\sigma } \left[ \partial_{\mu }g_{\sigma \nu }+\partial_{\nu }g_{\sigma \mu }- \partial_{\sigma }g_{\mu \nu }\right]
\end{equation}
from this I get contribution only for $\sigma =t$, because of how is defined the metric. 
\begin{equation}
	\left( \sigma = t \right) \to \Gamma ^{t}_{\mu \nu } = \frac{1}{2}g^{tt}\left[ \partial_{\mu }g_{t\nu } + \partial_{\nu }g_{t\mu } - \partial_{t}g_{\mu \nu }\right]
\end{equation}
At this point, one immediately see that the last derivative is null since staticness.\par
If $\mu , \nu \neq t \to 0$ since components of the metric tensor with mixed indices are null. 
\begin{equation}
\Gamma ^{t}_{tr} = \frac{1}{2} g^{tt}\partial_{r}g_{tt} = \frac{1}{2} \left( -e^{-2\alpha }\left( r \right) \right)\left( -2\alpha ' e^{+2\alpha \left( r \right)} \right) = \alpha ' \to \Gamma ^{t}_{tr} = \alpha '
\end{equation}
Now for $\rho =r$, so $\sigma $ must be \emph{r}, too:
\begin{equation}
	\Gamma ^{t}_{\mu \nu } = \frac{1}{2} g^{rr}\left[ \partial_{\mu }g_{r\nu }+ \partial_{\nu }g_{r\mu }-\partial_{r}g_{\mu \nu }\right]
\end{equation}
if  also$\mu ,\nu  = r$
\begin{equation}
	\Gamma ^{r}_{rr} = \frac{1}{2} g^{rr}\left[ \partial_{r}g_{rr}+\partial_{r}g_{rr}-\partial_{r}g_{rr}\right] = \frac{1}{2}g^{rr}d_{r}g_{rr} = \beta ' \to  \Gamma ^{r}_{rr} = \beta '
\end{equation}
instead, if $\mu ,\nu  = t$:
\begin{equation}
\Gamma^{r}_{tt} = \frac{1}{2}g^{rr} \left( \partial_{t}g_{rt} +\partial_{t}g_{rt}-\partial_{r}g_{tt} \right) = -\frac{1}{2} g^{rr}\partial_{r}g_{tt} = \frac{1}{2}e^{-2\beta }e^{2\alpha }2\alpha ' = \alpha ' e^{2\left( \alpha -\beta  \right)}
\end{equation}
The remaining components are given but it would be interesting retrieving them by yourself, since professor said that he could ask to compute one component of the $\Gamma $ at the exam.
\begin{equation}
\begin{matrix}
\Gamma ^{r}_{\theta \theta } = - r e^{-2\beta } & \Gamma ^{r}_{\phi \phi } = -r sin^{2}\theta e^{-2\beta } & \Gamma^{\theta }_{r\theta } = \frac{1}{r} \\
\Gamma ^{\theta }_{\phi \phi } = -sin \theta cos \theta  & \Gamma ^{\phi }_{r\phi } = \frac{1}{r} & \Gamma ^{\phi }_{\theta \phi } = \frac{cos \theta }{sin \theta } ,
\end{matrix} 
\end{equation}

\paragraph{Riemann tensor}
As we know it is defined as
\[
R^{\rho }_{\sigma \mu \nu } = \partial_{\mu }\Gamma ^{\rho }_{\nu \sigma }- \partial_{\nu }\Gamma ^{\rho }_{\mu \sigma } + \Gamma ^{\rho }_{\mu \lambda }\Gamma ^{\lambda }_{\nu \sigma }- \Gamma ^{\rho }_{\nu \lambda }\Gamma ^{\lambda }_{\mu \sigma }
\]
And since we know all the $\Gamma$s we should know every Riemann tensor component. Eventually we care about $R_{\mu \nu } = 0$, so we will impose
\[
R_{\alpha \alpha } = 0, \alpha \text{ fixed } \to R^{\beta }_{\alpha \beta \alpha }
\]
\begin{itemize}
\item $R^{t}_{rtr} = -\alpha^{\prime \prime } -\alpha ^{\prime 2} +\alpha^{\prime }\beta ^{\prime }$
\item $R^{t}_{\theta  t \theta } = -r \alpha^{\prime } e^{-2\beta }$
\item $R^{t}_{\phi t \phi } = -r \alpha^{\prime } sin^{2} \theta  e^{-2\beta }$
\item $R^{r}_{\theta r \theta } = r\beta ' e^{-2\beta }$
\item $R^{r}_{\phi r \phi } = r \beta ' sin^{2}\theta  e^{-2\beta }$
\item $R^{\theta }_{\phi \theta \phi } = \left( 1-e^{-2\beta } \right)sin^{2} \theta $
\end{itemize}
How to compute the others?
\begin{equation}
R^{r}_{trt} = g^{rr}R_{rtrt} = g^{rr}R_{trtr} = g^{rr}g_{tt}T^{t}_{rtr}
\end{equation}
So we can reconstruct them from elements we already have.

\paragraph{Ricci tensor}
We want $R_{\alpha \alpha } = 0$. Let's see:
\begin{equation}
	R_{tt} = R^{r}_{trt} + R^{\theta }_{t\theta t} + R^{\phi }_{t\phi t} = e^{2\left( \alpha -\beta  \right)}\left[ \alpha^{\prime \prime } + \alpha ^{\prime 2} - \alpha '\beta ' +\frac{2\alpha '}{r}\right] 
\end{equation}
\begin{equation}
	R_{\theta \theta } = \left[ \left( \beta ' - \alpha '  \right)r-1\right] e^{-2\beta }+1
\end{equation}
\begin{equation}
R_{rr} = -\alpha^{\prime \prime } - \alpha ^{\prime 2} + \alpha ' \beta ' + 2 \frac{\beta '}{r}
\end{equation}
\begin{equation}
	R_{\phi \phi } = R_{\theta \theta }sin^{2}\theta 
\end{equation}

They are individually equal to zero.

I will use some of these Ricci tensor components to retrieve some informations about $\alpha, \beta $.
\begin{gather*}
e^{2\left( \beta -\alpha  \right)}R_{tt} + R_{rr} = 0 \\
\frac{2}{r}\left( \alpha '-\beta ' \right) = 0 , r\neq 0 \\
\alpha '+ \beta ' = 0 \to  \alpha \left( r \right)+\beta \left( r \right) = const
\end{gather*}
it is possible to choose the time coordinates \emph{t} such that the constant is null.

Now we can do some restrictions on the metric
\begin{equation}
\begin{cases}
ds^{2} = -e ^{2\alpha \left( r \right)}dt^{2} + e^{2\beta \left( r \right)} dr^{2} + r^{2 d\Omega ^{2}} \\
 \alpha \left( r  \right) + \beta \left( r \right) = \gamma \to  \text{ constant }\\
\end{cases}
\end{equation}
this allows us to write
\begin{align}
	ds^{2} &= -e^{2\left( \gamma  - \beta \left( r \right) \right)} dt^{2} + e^{2\beta \left( r \right)} dr^{2} + r^{2} d\Omega^{2} = \\
	       & = -e ^{-2\beta \left( r \right)} \left( e^{\gamma }dt \right)^{2} + e^{2\beta\left( r \right) }dr^{2} + r^{2} d\Omega ^{2}  
\end{align}
we will operate this change of variables $dt^{\prime } = e^{\gamma }dt$.\par
Now I can set $\gamma =0$ and trying to fins the solution for $\alpha \left( r \right)$.
\[
\alpha \left( r \right) = - \beta \left( r \right)
\]
I use $R_{\theta \theta } = 0$ to find the missing function, solving for $\alpha $:
\begin{gather*}
	\left[ \left( -2\alpha \prime  \right)r-1\right]e^{2\alpha }+1 = 0 \\
	\frac{\partial }{\partial r} \left( re^{2\alpha } \right) = +1
\end{gather*}
this is a differential equation 
\begin{gather*}
re^{2\alpha }\left( r \right) = r + D \text{ , D is a constant } \\
e^{2\alpha \left( r \right)} = 1 + \frac{D}{r}
\end{gather*}
I could call \emph{D} radius, and introduce a new variable $R_{S} = - D$
\begin{equation}
ds^{2} = - \left( 1- \frac{R_{S}}{r} \right)dt^{2} + \left( 1 - \frac{R_{S}}{r} \right)^{-1} dr^{2} + r^{2}d\Omega ^{2}
\end{equation}
What is $R_{S}$? And what is his physical meaning? Newtonian gravitation helps us. This one is the most  general solution of spherical symmetry and it has to reproduce something we already know.\par

In the weak field regime we discussed how
\begin{equation}
\begin{cases}
g_{\mu \nu } = \eta _{\mu \nu } + h_{\mu \nu } \\
h_{00} = -2\Phi  \\
\end{cases}
\end{equation}
with $\Phi $ gravitational potential. 
\[
\Phi  =  - \frac{GM}{r}
\]
so by comparison
\[
R_{S} = 2GM
\]
We need to adjust some dimensions still
\[
\left[ G \right] \left[ M \right] = \left[ F \right]\left[ L^{2} \right]\left[ M^{-2} \right]\left[ M \right] = \left[ L \right]\left[ T^{-2} \right]\left[ L^{2} \right] = \left[ L^{3} \right]\left[ T^{-2} \right] 
\]
so if we divide for velocity squared we get a distance. This means
\[
R_{S } = \frac{2GM}{c^{2}}
\]
Now if one plugs in the values of the masses of Sun and Earth, he can find the relative $R_{S}$, that is something like 3km and 9mm. This is where the metric blows up.\par

\paragraph{Singularities}
The metric blows up for $r = R_{S}, 0$, but this shouldn't be a problem since the metric previously derived is valid well outside the Sun. It could be a problem if the object is bigger or if the radius of the object is smaller than the Schwarzschild radius,\par
What is a singularity?  Be in polar coordinates in the plane
\[
g^{\theta \theta } = \frac{1}{r^{2}}
\]
the only issue here is with the coordinates chosen, not with the space itself. But if I focus on \emph{scalars}, and I find one that blows up, then I have a singularity. Like
\[
R^{\mu \nu \rho \sigma }R_{\mu \nu \rho \sigma } = 48 \frac{G^{2}M^{2}}{r^{6}}
\]
we see that it blows up at $r=0$ but not at $r = R_{S}$





















