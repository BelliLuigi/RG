\chapter{Black Holes}
\section{Lec 16}
\subsection{Scharzschild's Metric}
The most obvious application of a theory of gravity is the case of spherical symmetry, that is like the one of the Earth or the Sun. We will start with a solution of the vacuum outside them, because is both easier and more useful. 
A specific solution of the Einstein Equation is the Schwarzschild Solution, for a static (components of $g_{\mu \nu }$ do not depend on time) spacetime with $\left( S^{2} \right)$ symmetry.\par
There is more than one way to derive this solution, we will use the most boring one:
\begin{enumerate}
\item \emph{Guess} a generic form for $g_{\mu \nu } $
\item Compute $\Gamma ^{\mu }_{\alpha \beta }$, $R^{\alpha }_{\beta \gamma \delta }$, $R_{\alpha \beta }$
\item Solve $R_{\mu \nu } = 0$, since the space outside the sun is empty space, vacuum.
\end{enumerate}

Let's start from the \textbf{guess} step.
We want to \emph{guess} the metric in spherical coordinates 
\begin{equation}\label{eq:polarmetric}
ds^{2}= -A\left( r \right) dt^{2} + B\left( r \right) dr^{2} + C\left( r \right) r^{2}d\Omega ^{2}
\end{equation}

where $d\Omega ^{2}$ is the metric on a $S^{2}$ sphere.
\[
	d\Omega ^{2} = d\theta ^{2} + sin^{2}\theta d\phi ^{2}
\]
Since we are almost in the vacuum, we can start in fact from the Minkowski metric in polar coordinates: $ds^{2 }=-dt^{2} +dr^{2}+r^{2}d\Omega ^{2} $.
Now we would like simplify a little eq.\ref{eq:polarmetric}:
\begin{itemize}
\item $r^{2}C\left( r \right) \to  r^{2}$, this is a change of coordinates, nothing to do with spherical symmetry, but know we have to rescale $A\left( r \right), B\left( r \right)$
\item $A \equiv e^{2\alpha \left( r \right)}$
\item $B\equive^{2\beta \left( r \right)}$
\end{itemize}.
With this the most general guess is:
\begin{equation}\label{eq:pmetric2}
ds^{2} = -e^{2\alpha \left( r \right)}dt^{2} + e^{2\beta \left( r \right)}dr^{2} + r^{2}d\Omega ^{2}
\end{equation}
We have now two unknown functions $\alpha ,\beta $ of the radial coordinate, not time because of staticness, not angle because of isotropy.
	

