\chapter{Black Holes}
\section{Lec 16}
\subsection{Scharzschild's Metric}
The most obvious application of a theory of gravity is the case of spherical symmetry, that is like the one of the Earth or the Sun. We will start with a solution of the vacuum outside them, because is both easier and more useful. 
A specific solution of the Einstein Equation is the Schwarzschild Solution, for a static (components of $g_{\mu \nu }$ do not depend on time) spacetime with $\left( S^{2} \right)$ symmetry.\par
There is more than one way to derive this solution, we will use the most boring one:
\begin{enumerate}
\item \emph{Guess} a generic form for $g_{\mu \nu } $
\item Compute $\Gamma ^{\mu }_{\alpha \beta }$, $R^{\alpha }_{\beta \gamma \delta }$, $R_{\alpha \beta }$
\item Solve $R_{\mu \nu } = 0$, since the space outside the sun is empty space, vacuum.
\end{enumerate}

Let's start from the \textbf{guess} step.
We want to \emph{guess} the metric in spherical coordinates 
\begin{equation}\label{eq:polarmetric}
ds^{2}= -A\left( r \right) dt^{2} + B\left( r \right) dr^{2} + C\left( r \right) r^{2}d\Omega ^{2}
\end{equation}

where $d\Omega ^{2}$ is the metric on a $S^{2}$ sphere.
\[
	d\Omega ^{2} = d\theta ^{2} + sin^{2}\theta d\phi ^{2}
\]
Since we are almost in the vacuum, we can start in fact from the Minkowski metric in polar coordinates: $ds^{2 }=-dt^{2} +dr^{2}+r^{2}d\Omega ^{2} $.
Now we would like simplify a little eq.\ref{eq:polarmetric}:
\begin{itemize}
\item $r^{2}C\left( r \right) \to  r^{2}$, this is a change of coordinates, nothing to do with spherical symmetry, but know we have to rescale $A\left( r \right), B\left( r \right)$
\item $A \equiv e^{2\alpha \left( r \right)}$
\item $B\equiv e^{2\beta \left( r \right)}$
\end{itemize}.
With this, the most general guess is:
\begin{equation}\label{eq:pmetric2}
ds^{2} = -e^{2\alpha \left( r \right)}dt^{2} + e^{2\beta \left( r \right)}dr^{2} + r^{2}d\Omega ^{2}
\end{equation}
We have now two unknown functions $\alpha ,\beta $ of the radial coordinate, not time because of staticness, not angle because of isotropy.\par
Now it's time to \textbf{compute the metric tensor, Christoffel thing and Riemann tensor.}\par
\paragraph{Metric} 
\begin{equation}
\begin{matrix} 
g_{tt}= -e^{2\alpha \left( r \right)} & g_{rr}=e^{2\beta \left( r \right)} & g_{\theta \theta }= r^{2} & g_{\phi \phi } = r^{2}sin^{2}\theta  \\
g^{tt}=-e^{-2\alpha \left( r \right)} & g^{rr}=e^{-2\beta \left( r \right)} & g^{\theta \theta }= \frac{1}{r^{2}} & g^{\phi \phi }=\frac{1}{r^{2}sin^{2}\theta } \\
\end{matrix}
\end{equation}
other entries are null.
\paragraph{Christoffel Symbols}
 In most generic spacetimes there are 40 independent Christoffel coefficients. %unclear why there are less
As you may remember the Christoffel connection is defined as
\begin{equation}
	\Gamma ^{\rho }_{\mu \nu } = \frac{1}{2}g^{\rho \sigma }\left[ \partial_{\mu }g_{\sigma \nu }+\partial_{\nu }g_{\sigma \mu }-\partial_{\sigma }g_{\mu \nu }\right]
\end{equation}
if $\rho = t $:
\begin{equation}
	\Gamma^{t}_{\mu \nu } = \frac{1}{2} g^{t\sigma } \left[ \partial_{\mu }g_{\sigma \nu }+\partial_{\nu }g_{\sigma \mu }- \partial_{\sigma }g_{\mu \nu }\right]
\end{equation}
from this I get contribution only for $\sigma =t$, because of how is defined the metric. 
\begin{equation}
	\left( \sigma = t \right) \to \Gamma ^{t}_{\mu \nu } = \frac{1}{2}g^{tt}\left[ \partial_{\mu }g_{t\nu } + \partial_{\nu }g_{t\mu } - \partial_{t}g_{\mu \nu }\right]
\end{equation}
At this point, one immediately see that the last derivative is null since staticness.\par
If $\mu , \nu \neq t \to 0$ since components of the metric tensor with mixed indices are null. 
\begin{equation}
\Gamma ^{t}_{tr} = \frac{1}{2} g^{tt}\partial_{r}g_{tt} = \frac{1}{2} \left( -e^{-2\alpha }\left( r \right) \right)\left( -2\alpha ' e^{+2\alpha \left( r \right)} \right) = \alpha ' \to \Gamma ^{t}_{tr} = \alpha '
\end{equation}
Now for $\rho =r$, so $\sigma $ must be \emph{r}, too:
\begin{equation}
	\Gamma ^{t}_{\mu \nu } = \frac{1}{2} g^{rr}\left[ \partial_{\mu }g_{r\nu }+ \partial_{\nu }g_{r\mu }-\partial_{r}g_{\mu \nu }\right]
\end{equation}
if  also$\mu ,\nu  = r$
\begin{equation}
	\Gamma ^{r}_{rr} = \frac{1}{2} g^{rr}\left[ \partial_{r}g_{rr}+\partial_{r}g_{rr}-\partial_{r}g_{rr}\right] = \frac{1}{2}g^{rr}d_{r}g_{rr} = \beta ' \to  \Gamma ^{r}_{rr} = \beta '
\end{equation}
instead, if $\mu ,\nu  = t$:
\begin{equation}
\Gamma^{r}_{tt} = \frac{1}{2}g^{rr} \left( \partial_{t}g_{rt} +\partial_{t}g_{rt}-\partial_{r}g_{tt} \right) = -\frac{1}{2} g^{rr}\partial_{r}g_{tt} = \frac{1}{2}e^{-2\beta }e^{2\alpha }2\alpha ' = \alpha ' e^{2\left( \alpha -\beta  \right)}
\end{equation}
The remaining components are given but it would be interesting retrieving them by yourself, since professor said that he could ask to compute one component of the $\Gamma $ at the exam.
\begin{equation}
\begin{matrix}
\Gamma ^{r}_{\theta \theta } = - r e^{-2\beta } & \Gamma ^{r}_{\phi \phi } = -r sin^{2}\theta e^{-2\beta } & \Gamma^{\theta }_{r\theta } = \frac{1}{r} \\
\Gamma ^{\theta }_{\phi \phi } = -sin \theta cos \theta  & \Gamma ^{\phi }_{r\phi } = \frac{1}{r} & \Gamma ^{\phi }_{\theta \phi } = \frac{cos \theta }{sin \theta } ,
\end{matrix} 
\end{equation}

\paragraph{Riemann tensor}
As we know it is defined as
\[
R^{\rho }_{\sigma \mu \nu } = \partial_{\mu }\Gamma ^{\rho }_{\nu \sigma }- \partial_{\nu }\Gamma ^{\rho }_{\mu \sigma } + \Gamma ^{\rho }_{\mu \lambda }\Gamma ^{\lambda }_{\nu \sigma }- \Gamma ^{\rho }_{\nu \lambda }\Gamma ^{\lambda }_{\mu \sigma }
\]
And since we know all the $\Gamma$s we should know every Riemann tensor component. Eventually we care about $R_{\mu \nu } = 0$, so we will impose
\[
R_{\alpha \alpha } = 0, \alpha \text{ fixed } \to R^{\beta }_{\alpha \beta \alpha }
\]
\begin{itemize}
\item $R^{t}_{rtr} = -\alpha '' -\alpha ^{2}' +\alpha '\beta '$
\item $R^{t}_{\theta  t \theta } = -r \alpha ' e^{-2\beta }$
\item $R^{t}_{\phi t \phi } = -r \alpha ' sin^{2} \theta  e^{-2\beta }$
\item $R^{r}_{\theta r \theta } = r\beta ' e^{-2\beta }$
\item $R^{r}_{\phi r \phi } = r \beta ' sin^{2}\theta  e^{-2\beta }$
\item $R^{\theta }_{\phi \theta \phi } = \left( 1-e^{-2\beta } \right)sin^{2} \theta $
\end{itemize}
How to compute the others?
\begin{equation}
R^{r}_{trt} = g^{rr}R_{rtrt} = g^{rr}R_{trtr} = g^{rr}g_{tt}T^{t}_{rtr}
\end{equation}
So we can reconstruct them from elements we already have.

\paragraph{Ricci tensor}
We want $R_{\alpha \alpha } = 0$. Let's see:
\begin{equation}
	R_{tt} = R^{r}_{trt} + R^{\theta }_{t\theta t} + R^{\phi }_{t\phi t} = e^{2\left( \alpha -\beta  \right)}\left[ \alpha '' + \alpha ^{2}' - \alpha '\beta ' +\frac{2\alpha '}{r}\right] 
\end{equation}
\begin{equation}
	R_{\theta \theta } = \left[ \left( \beta ' - \alpha '  \right)r-1\right] e^{-2\beta }+1
\end{equation}
\begin{equation}
R_{rr} = -\alpha '' - \alpha ^{2}' + \alpha ' \beta ' + 2 \frac{\beta '}{r}
\end{equation}
\begin{equation}
	R_{\phi \phi } = R_{\theta \theta }sin^{2}\theta 
\end{equation}

They are individually equal to zero.

I will use some of these Ricci tensor components to retrieve some informations about $\alpha, \beta $.
\begin{gather*}
e^{2\left( \beta -\alpha  \right)}R_{tt} + R_{rr} = 0 \\
\frac{2}{r}\left( \alpha '-\beta ' \right) = 0 , r\neq 0 \\
\alpha '+ \beta ' = 0 \to  \alpha \left( r \right)+\beta \left( r \right) = const
\end{gather*}
it is possible to choose the time coordinates \emph{t} such that the constant is null.






















